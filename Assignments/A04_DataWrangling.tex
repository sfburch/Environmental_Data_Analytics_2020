\PassOptionsToPackage{unicode=true}{hyperref} % options for packages loaded elsewhere
\PassOptionsToPackage{hyphens}{url}
%
\documentclass[]{article}
\usepackage{lmodern}
\usepackage{amssymb,amsmath}
\usepackage{ifxetex,ifluatex}
\usepackage{fixltx2e} % provides \textsubscript
\ifnum 0\ifxetex 1\fi\ifluatex 1\fi=0 % if pdftex
  \usepackage[T1]{fontenc}
  \usepackage[utf8]{inputenc}
  \usepackage{textcomp} % provides euro and other symbols
\else % if luatex or xelatex
  \usepackage{unicode-math}
  \defaultfontfeatures{Ligatures=TeX,Scale=MatchLowercase}
\fi
% use upquote if available, for straight quotes in verbatim environments
\IfFileExists{upquote.sty}{\usepackage{upquote}}{}
% use microtype if available
\IfFileExists{microtype.sty}{%
\usepackage[]{microtype}
\UseMicrotypeSet[protrusion]{basicmath} % disable protrusion for tt fonts
}{}
\IfFileExists{parskip.sty}{%
\usepackage{parskip}
}{% else
\setlength{\parindent}{0pt}
\setlength{\parskip}{6pt plus 2pt minus 1pt}
}
\usepackage{hyperref}
\hypersetup{
            pdftitle={Assignment 4: Data Wrangling},
            pdfauthor={Samantha Burch},
            pdfborder={0 0 0},
            breaklinks=true}
\urlstyle{same}  % don't use monospace font for urls
\usepackage[margin=2.54cm]{geometry}
\usepackage{color}
\usepackage{fancyvrb}
\newcommand{\VerbBar}{|}
\newcommand{\VERB}{\Verb[commandchars=\\\{\}]}
\DefineVerbatimEnvironment{Highlighting}{Verbatim}{commandchars=\\\{\}}
% Add ',fontsize=\small' for more characters per line
\usepackage{framed}
\definecolor{shadecolor}{RGB}{248,248,248}
\newenvironment{Shaded}{\begin{snugshade}}{\end{snugshade}}
\newcommand{\AlertTok}[1]{\textcolor[rgb]{0.94,0.16,0.16}{#1}}
\newcommand{\AnnotationTok}[1]{\textcolor[rgb]{0.56,0.35,0.01}{\textbf{\textit{#1}}}}
\newcommand{\AttributeTok}[1]{\textcolor[rgb]{0.77,0.63,0.00}{#1}}
\newcommand{\BaseNTok}[1]{\textcolor[rgb]{0.00,0.00,0.81}{#1}}
\newcommand{\BuiltInTok}[1]{#1}
\newcommand{\CharTok}[1]{\textcolor[rgb]{0.31,0.60,0.02}{#1}}
\newcommand{\CommentTok}[1]{\textcolor[rgb]{0.56,0.35,0.01}{\textit{#1}}}
\newcommand{\CommentVarTok}[1]{\textcolor[rgb]{0.56,0.35,0.01}{\textbf{\textit{#1}}}}
\newcommand{\ConstantTok}[1]{\textcolor[rgb]{0.00,0.00,0.00}{#1}}
\newcommand{\ControlFlowTok}[1]{\textcolor[rgb]{0.13,0.29,0.53}{\textbf{#1}}}
\newcommand{\DataTypeTok}[1]{\textcolor[rgb]{0.13,0.29,0.53}{#1}}
\newcommand{\DecValTok}[1]{\textcolor[rgb]{0.00,0.00,0.81}{#1}}
\newcommand{\DocumentationTok}[1]{\textcolor[rgb]{0.56,0.35,0.01}{\textbf{\textit{#1}}}}
\newcommand{\ErrorTok}[1]{\textcolor[rgb]{0.64,0.00,0.00}{\textbf{#1}}}
\newcommand{\ExtensionTok}[1]{#1}
\newcommand{\FloatTok}[1]{\textcolor[rgb]{0.00,0.00,0.81}{#1}}
\newcommand{\FunctionTok}[1]{\textcolor[rgb]{0.00,0.00,0.00}{#1}}
\newcommand{\ImportTok}[1]{#1}
\newcommand{\InformationTok}[1]{\textcolor[rgb]{0.56,0.35,0.01}{\textbf{\textit{#1}}}}
\newcommand{\KeywordTok}[1]{\textcolor[rgb]{0.13,0.29,0.53}{\textbf{#1}}}
\newcommand{\NormalTok}[1]{#1}
\newcommand{\OperatorTok}[1]{\textcolor[rgb]{0.81,0.36,0.00}{\textbf{#1}}}
\newcommand{\OtherTok}[1]{\textcolor[rgb]{0.56,0.35,0.01}{#1}}
\newcommand{\PreprocessorTok}[1]{\textcolor[rgb]{0.56,0.35,0.01}{\textit{#1}}}
\newcommand{\RegionMarkerTok}[1]{#1}
\newcommand{\SpecialCharTok}[1]{\textcolor[rgb]{0.00,0.00,0.00}{#1}}
\newcommand{\SpecialStringTok}[1]{\textcolor[rgb]{0.31,0.60,0.02}{#1}}
\newcommand{\StringTok}[1]{\textcolor[rgb]{0.31,0.60,0.02}{#1}}
\newcommand{\VariableTok}[1]{\textcolor[rgb]{0.00,0.00,0.00}{#1}}
\newcommand{\VerbatimStringTok}[1]{\textcolor[rgb]{0.31,0.60,0.02}{#1}}
\newcommand{\WarningTok}[1]{\textcolor[rgb]{0.56,0.35,0.01}{\textbf{\textit{#1}}}}
\usepackage{graphicx,grffile}
\makeatletter
\def\maxwidth{\ifdim\Gin@nat@width>\linewidth\linewidth\else\Gin@nat@width\fi}
\def\maxheight{\ifdim\Gin@nat@height>\textheight\textheight\else\Gin@nat@height\fi}
\makeatother
% Scale images if necessary, so that they will not overflow the page
% margins by default, and it is still possible to overwrite the defaults
% using explicit options in \includegraphics[width, height, ...]{}
\setkeys{Gin}{width=\maxwidth,height=\maxheight,keepaspectratio}
\setlength{\emergencystretch}{3em}  % prevent overfull lines
\providecommand{\tightlist}{%
  \setlength{\itemsep}{0pt}\setlength{\parskip}{0pt}}
\setcounter{secnumdepth}{0}
% Redefines (sub)paragraphs to behave more like sections
\ifx\paragraph\undefined\else
\let\oldparagraph\paragraph
\renewcommand{\paragraph}[1]{\oldparagraph{#1}\mbox{}}
\fi
\ifx\subparagraph\undefined\else
\let\oldsubparagraph\subparagraph
\renewcommand{\subparagraph}[1]{\oldsubparagraph{#1}\mbox{}}
\fi

% set default figure placement to htbp
\makeatletter
\def\fps@figure{htbp}
\makeatother


\title{Assignment 4: Data Wrangling}
\author{Samantha Burch}
\date{}

\begin{document}
\maketitle

\hypertarget{overview}{%
\subsection{OVERVIEW}\label{overview}}

This exercise accompanies the lessons in Environmental Data Analytics on
Data Wrangling

\hypertarget{directions}{%
\subsection{Directions}\label{directions}}

\begin{enumerate}
\def\labelenumi{\arabic{enumi}.}
\tightlist
\item
  Change ``Student Name'' on line 3 (above) with your name.
\item
  Work through the steps, \textbf{creating code and output} that fulfill
  each instruction.
\item
  Be sure to \textbf{answer the questions} in this assignment document.
\item
  When you have completed the assignment, \textbf{Knit} the text and
  code into a single PDF file.
\item
  After Knitting, submit the completed exercise (PDF file) to the
  dropbox in Sakai. Add your last name into the file name (e.g.,
  ``Salk\_A04\_DataWrangling.Rmd'') prior to submission.
\end{enumerate}

The completed exercise is due on Tuesday, February 4 at 1:00 pm.

\hypertarget{set-up-your-session}{%
\subsection{Set up your session}\label{set-up-your-session}}

\begin{enumerate}
\def\labelenumi{\arabic{enumi}.}
\item
  Check your working directory, load the \texttt{tidyverse} and
  \texttt{lubridate} packages, and upload all four raw data files
  associated with the EPA Air dataset. See the README file for the EPA
  air datasets for more information (especially if you have not worked
  with air quality data previously).
\item
  Explore the dimensions, column names, and structure of the datasets.
\end{enumerate}

\begin{Shaded}
\begin{Highlighting}[]
\CommentTok{#1}
\CommentTok{#load tidyverse and lubridate packages}
\KeywordTok{library}\NormalTok{(tidyverse)}
\KeywordTok{library}\NormalTok{(lubridate)}

\CommentTok{#import the 4 raw data files}
\NormalTok{EPAAir1 <-}\StringTok{ }\KeywordTok{read.csv}\NormalTok{(}\StringTok{"./Data/Raw/EPAair_O3_NC2018_raw.csv"}\NormalTok{)}
\NormalTok{EPAAir2 <-}\StringTok{ }\KeywordTok{read.csv}\NormalTok{(}\StringTok{"./Data/Raw/EPAair_O3_NC2019_raw.csv"}\NormalTok{)}
\NormalTok{EPAAir3 <-}\StringTok{ }\KeywordTok{read.csv}\NormalTok{(}\StringTok{"./Data/Raw/EPAair_PM25_NC2018_raw.csv"}\NormalTok{)}
\NormalTok{EPAAir4 <-}\StringTok{ }\KeywordTok{read.csv}\NormalTok{(}\StringTok{"./Data/Raw/EPAair_PM25_NC2019_raw.csv"}\NormalTok{)}

\CommentTok{#2}
\CommentTok{#explore the dimensions}
\KeywordTok{view}\NormalTok{(EPAAir1)}
\KeywordTok{dim}\NormalTok{(EPAAir1)}
\end{Highlighting}
\end{Shaded}

\begin{verbatim}
## [1] 9737   20
\end{verbatim}

\begin{Shaded}
\begin{Highlighting}[]
\KeywordTok{dim}\NormalTok{(EPAAir2)}
\end{Highlighting}
\end{Shaded}

\begin{verbatim}
## [1] 10592    20
\end{verbatim}

\begin{Shaded}
\begin{Highlighting}[]
\KeywordTok{dim}\NormalTok{(EPAAir3)}
\end{Highlighting}
\end{Shaded}

\begin{verbatim}
## [1] 8983   20
\end{verbatim}

\begin{Shaded}
\begin{Highlighting}[]
\KeywordTok{dim}\NormalTok{(EPAAir4)}
\end{Highlighting}
\end{Shaded}

\begin{verbatim}
## [1] 8581   20
\end{verbatim}

\begin{Shaded}
\begin{Highlighting}[]
\CommentTok{#explore column names}
\KeywordTok{colnames}\NormalTok{(EPAAir1)}
\end{Highlighting}
\end{Shaded}

\begin{verbatim}
##  [1] "Date"                                
##  [2] "Source"                              
##  [3] "Site.ID"                             
##  [4] "POC"                                 
##  [5] "Daily.Max.8.hour.Ozone.Concentration"
##  [6] "UNITS"                               
##  [7] "DAILY_AQI_VALUE"                     
##  [8] "Site.Name"                           
##  [9] "DAILY_OBS_COUNT"                     
## [10] "PERCENT_COMPLETE"                    
## [11] "AQS_PARAMETER_CODE"                  
## [12] "AQS_PARAMETER_DESC"                  
## [13] "CBSA_CODE"                           
## [14] "CBSA_NAME"                           
## [15] "STATE_CODE"                          
## [16] "STATE"                               
## [17] "COUNTY_CODE"                         
## [18] "COUNTY"                              
## [19] "SITE_LATITUDE"                       
## [20] "SITE_LONGITUDE"
\end{verbatim}

\begin{Shaded}
\begin{Highlighting}[]
\KeywordTok{colnames}\NormalTok{(EPAAir2)}
\end{Highlighting}
\end{Shaded}

\begin{verbatim}
##  [1] "Date"                                
##  [2] "Source"                              
##  [3] "Site.ID"                             
##  [4] "POC"                                 
##  [5] "Daily.Max.8.hour.Ozone.Concentration"
##  [6] "UNITS"                               
##  [7] "DAILY_AQI_VALUE"                     
##  [8] "Site.Name"                           
##  [9] "DAILY_OBS_COUNT"                     
## [10] "PERCENT_COMPLETE"                    
## [11] "AQS_PARAMETER_CODE"                  
## [12] "AQS_PARAMETER_DESC"                  
## [13] "CBSA_CODE"                           
## [14] "CBSA_NAME"                           
## [15] "STATE_CODE"                          
## [16] "STATE"                               
## [17] "COUNTY_CODE"                         
## [18] "COUNTY"                              
## [19] "SITE_LATITUDE"                       
## [20] "SITE_LONGITUDE"
\end{verbatim}

\begin{Shaded}
\begin{Highlighting}[]
\KeywordTok{colnames}\NormalTok{(EPAAir3)}
\end{Highlighting}
\end{Shaded}

\begin{verbatim}
##  [1] "Date"                           "Source"                        
##  [3] "Site.ID"                        "POC"                           
##  [5] "Daily.Mean.PM2.5.Concentration" "UNITS"                         
##  [7] "DAILY_AQI_VALUE"                "Site.Name"                     
##  [9] "DAILY_OBS_COUNT"                "PERCENT_COMPLETE"              
## [11] "AQS_PARAMETER_CODE"             "AQS_PARAMETER_DESC"            
## [13] "CBSA_CODE"                      "CBSA_NAME"                     
## [15] "STATE_CODE"                     "STATE"                         
## [17] "COUNTY_CODE"                    "COUNTY"                        
## [19] "SITE_LATITUDE"                  "SITE_LONGITUDE"
\end{verbatim}

\begin{Shaded}
\begin{Highlighting}[]
\KeywordTok{colnames}\NormalTok{(EPAAir4)}
\end{Highlighting}
\end{Shaded}

\begin{verbatim}
##  [1] "Date"                           "Source"                        
##  [3] "Site.ID"                        "POC"                           
##  [5] "Daily.Mean.PM2.5.Concentration" "UNITS"                         
##  [7] "DAILY_AQI_VALUE"                "Site.Name"                     
##  [9] "DAILY_OBS_COUNT"                "PERCENT_COMPLETE"              
## [11] "AQS_PARAMETER_CODE"             "AQS_PARAMETER_DESC"            
## [13] "CBSA_CODE"                      "CBSA_NAME"                     
## [15] "STATE_CODE"                     "STATE"                         
## [17] "COUNTY_CODE"                    "COUNTY"                        
## [19] "SITE_LATITUDE"                  "SITE_LONGITUDE"
\end{verbatim}

\begin{Shaded}
\begin{Highlighting}[]
\CommentTok{#explore the strucutre}
\KeywordTok{str}\NormalTok{(EPAAir1)}
\end{Highlighting}
\end{Shaded}

\begin{verbatim}
## 'data.frame':    9737 obs. of  20 variables:
##  $ Date                                : Factor w/ 364 levels "01/01/2018","01/02/2018",..: 60 61 62 63 64 65 66 67 68 69 ...
##  $ Source                              : Factor w/ 1 level "AQS": 1 1 1 1 1 1 1 1 1 1 ...
##  $ Site.ID                             : int  370030005 370030005 370030005 370030005 370030005 370030005 370030005 370030005 370030005 370030005 ...
##  $ POC                                 : int  1 1 1 1 1 1 1 1 1 1 ...
##  $ Daily.Max.8.hour.Ozone.Concentration: num  0.043 0.046 0.047 0.049 0.047 0.03 0.036 0.044 0.049 0.043 ...
##  $ UNITS                               : Factor w/ 1 level "ppm": 1 1 1 1 1 1 1 1 1 1 ...
##  $ DAILY_AQI_VALUE                     : int  40 43 44 45 44 28 33 41 45 40 ...
##  $ Site.Name                           : Factor w/ 40 levels "","Beaufort",..: 35 35 35 35 35 35 35 35 35 35 ...
##  $ DAILY_OBS_COUNT                     : int  17 17 17 17 17 17 17 17 17 17 ...
##  $ PERCENT_COMPLETE                    : num  100 100 100 100 100 100 100 100 100 100 ...
##  $ AQS_PARAMETER_CODE                  : int  44201 44201 44201 44201 44201 44201 44201 44201 44201 44201 ...
##  $ AQS_PARAMETER_DESC                  : Factor w/ 1 level "Ozone": 1 1 1 1 1 1 1 1 1 1 ...
##  $ CBSA_CODE                           : int  25860 25860 25860 25860 25860 25860 25860 25860 25860 25860 ...
##  $ CBSA_NAME                           : Factor w/ 17 levels "","Asheville, NC",..: 9 9 9 9 9 9 9 9 9 9 ...
##  $ STATE_CODE                          : int  37 37 37 37 37 37 37 37 37 37 ...
##  $ STATE                               : Factor w/ 1 level "North Carolina": 1 1 1 1 1 1 1 1 1 1 ...
##  $ COUNTY_CODE                         : int  3 3 3 3 3 3 3 3 3 3 ...
##  $ COUNTY                              : Factor w/ 32 levels "Alexander","Avery",..: 1 1 1 1 1 1 1 1 1 1 ...
##  $ SITE_LATITUDE                       : num  35.9 35.9 35.9 35.9 35.9 ...
##  $ SITE_LONGITUDE                      : num  -81.2 -81.2 -81.2 -81.2 -81.2 ...
\end{verbatim}

\begin{Shaded}
\begin{Highlighting}[]
\KeywordTok{str}\NormalTok{(EPAAir2)}
\end{Highlighting}
\end{Shaded}

\begin{verbatim}
## 'data.frame':    10592 obs. of  20 variables:
##  $ Date                                : Factor w/ 365 levels "01/01/2019","01/02/2019",..: 1 2 3 4 5 6 7 8 9 10 ...
##  $ Source                              : Factor w/ 2 levels "AirNow","AQS": 1 1 1 1 1 1 1 1 1 1 ...
##  $ Site.ID                             : int  370030005 370030005 370030005 370030005 370030005 370030005 370030005 370030005 370030005 370030005 ...
##  $ POC                                 : int  1 1 1 1 1 1 1 1 1 1 ...
##  $ Daily.Max.8.hour.Ozone.Concentration: num  0.029 0.018 0.016 0.022 0.037 0.037 0.029 0.038 0.038 0.03 ...
##  $ UNITS                               : Factor w/ 1 level "ppm": 1 1 1 1 1 1 1 1 1 1 ...
##  $ DAILY_AQI_VALUE                     : int  27 17 15 20 34 34 27 35 35 28 ...
##  $ Site.Name                           : Factor w/ 38 levels "","Beaufort",..: 33 33 33 33 33 33 33 33 33 33 ...
##  $ DAILY_OBS_COUNT                     : int  24 24 24 24 24 24 24 24 24 24 ...
##  $ PERCENT_COMPLETE                    : num  100 100 100 100 100 100 100 100 100 100 ...
##  $ AQS_PARAMETER_CODE                  : int  44201 44201 44201 44201 44201 44201 44201 44201 44201 44201 ...
##  $ AQS_PARAMETER_DESC                  : Factor w/ 1 level "Ozone": 1 1 1 1 1 1 1 1 1 1 ...
##  $ CBSA_CODE                           : int  25860 25860 25860 25860 25860 25860 25860 25860 25860 25860 ...
##  $ CBSA_NAME                           : Factor w/ 15 levels "","Asheville, NC",..: 8 8 8 8 8 8 8 8 8 8 ...
##  $ STATE_CODE                          : int  37 37 37 37 37 37 37 37 37 37 ...
##  $ STATE                               : Factor w/ 1 level "North Carolina": 1 1 1 1 1 1 1 1 1 1 ...
##  $ COUNTY_CODE                         : int  3 3 3 3 3 3 3 3 3 3 ...
##  $ COUNTY                              : Factor w/ 30 levels "Alexander","Avery",..: 1 1 1 1 1 1 1 1 1 1 ...
##  $ SITE_LATITUDE                       : num  35.9 35.9 35.9 35.9 35.9 ...
##  $ SITE_LONGITUDE                      : num  -81.2 -81.2 -81.2 -81.2 -81.2 ...
\end{verbatim}

\begin{Shaded}
\begin{Highlighting}[]
\KeywordTok{str}\NormalTok{(EPAAir3)}
\end{Highlighting}
\end{Shaded}

\begin{verbatim}
## 'data.frame':    8983 obs. of  20 variables:
##  $ Date                          : Factor w/ 365 levels "01/01/2018","01/02/2018",..: 2 5 8 11 14 17 20 23 26 29 ...
##  $ Source                        : Factor w/ 1 level "AQS": 1 1 1 1 1 1 1 1 1 1 ...
##  $ Site.ID                       : int  370110002 370110002 370110002 370110002 370110002 370110002 370110002 370110002 370110002 370110002 ...
##  $ POC                           : int  1 1 1 1 1 1 1 1 1 1 ...
##  $ Daily.Mean.PM2.5.Concentration: num  2.9 3.7 5.3 0.8 2.5 4.5 1.8 2.5 4.2 1.7 ...
##  $ UNITS                         : Factor w/ 1 level "ug/m3 LC": 1 1 1 1 1 1 1 1 1 1 ...
##  $ DAILY_AQI_VALUE               : int  12 15 22 3 10 19 8 10 18 7 ...
##  $ Site.Name                     : Factor w/ 25 levels "","Blackstone",..: 15 15 15 15 15 15 15 15 15 15 ...
##  $ DAILY_OBS_COUNT               : int  1 1 1 1 1 1 1 1 1 1 ...
##  $ PERCENT_COMPLETE              : num  100 100 100 100 100 100 100 100 100 100 ...
##  $ AQS_PARAMETER_CODE            : int  88502 88502 88502 88502 88502 88502 88502 88502 88502 88502 ...
##  $ AQS_PARAMETER_DESC            : Factor w/ 2 levels "Acceptable PM2.5 AQI & Speciation Mass",..: 1 1 1 1 1 1 1 1 1 1 ...
##  $ CBSA_CODE                     : int  NA NA NA NA NA NA NA NA NA NA ...
##  $ CBSA_NAME                     : Factor w/ 14 levels "","Asheville, NC",..: 1 1 1 1 1 1 1 1 1 1 ...
##  $ STATE_CODE                    : int  37 37 37 37 37 37 37 37 37 37 ...
##  $ STATE                         : Factor w/ 1 level "North Carolina": 1 1 1 1 1 1 1 1 1 1 ...
##  $ COUNTY_CODE                   : int  11 11 11 11 11 11 11 11 11 11 ...
##  $ COUNTY                        : Factor w/ 21 levels "Avery","Buncombe",..: 1 1 1 1 1 1 1 1 1 1 ...
##  $ SITE_LATITUDE                 : num  36 36 36 36 36 ...
##  $ SITE_LONGITUDE                : num  -81.9 -81.9 -81.9 -81.9 -81.9 ...
\end{verbatim}

\begin{Shaded}
\begin{Highlighting}[]
\KeywordTok{str}\NormalTok{(EPAAir4)}
\end{Highlighting}
\end{Shaded}

\begin{verbatim}
## 'data.frame':    8581 obs. of  20 variables:
##  $ Date                          : Factor w/ 365 levels "01/01/2019","01/02/2019",..: 3 6 9 12 15 18 21 24 27 30 ...
##  $ Source                        : Factor w/ 2 levels "AirNow","AQS": 2 2 2 2 2 2 2 2 2 2 ...
##  $ Site.ID                       : int  370110002 370110002 370110002 370110002 370110002 370110002 370110002 370110002 370110002 370110002 ...
##  $ POC                           : int  1 1 1 1 1 1 1 1 1 1 ...
##  $ Daily.Mean.PM2.5.Concentration: num  1.6 1 1.3 6.3 2.6 1.2 1.5 1.5 3.7 1.6 ...
##  $ UNITS                         : Factor w/ 1 level "ug/m3 LC": 1 1 1 1 1 1 1 1 1 1 ...
##  $ DAILY_AQI_VALUE               : int  7 4 5 26 11 5 6 6 15 7 ...
##  $ Site.Name                     : Factor w/ 25 levels "","Board Of Ed. Bldg.",..: 14 14 14 14 14 14 14 14 14 14 ...
##  $ DAILY_OBS_COUNT               : int  1 1 1 1 1 1 1 1 1 1 ...
##  $ PERCENT_COMPLETE              : num  100 100 100 100 100 100 100 100 100 100 ...
##  $ AQS_PARAMETER_CODE            : int  88502 88502 88502 88502 88502 88502 88502 88502 88502 88502 ...
##  $ AQS_PARAMETER_DESC            : Factor w/ 2 levels "Acceptable PM2.5 AQI & Speciation Mass",..: 1 1 1 1 1 1 1 1 1 1 ...
##  $ CBSA_CODE                     : int  NA NA NA NA NA NA NA NA NA NA ...
##  $ CBSA_NAME                     : Factor w/ 14 levels "","Asheville, NC",..: 1 1 1 1 1 1 1 1 1 1 ...
##  $ STATE_CODE                    : int  37 37 37 37 37 37 37 37 37 37 ...
##  $ STATE                         : Factor w/ 1 level "North Carolina": 1 1 1 1 1 1 1 1 1 1 ...
##  $ COUNTY_CODE                   : int  11 11 11 11 11 11 11 11 11 11 ...
##  $ COUNTY                        : Factor w/ 21 levels "Avery","Buncombe",..: 1 1 1 1 1 1 1 1 1 1 ...
##  $ SITE_LATITUDE                 : num  36 36 36 36 36 ...
##  $ SITE_LONGITUDE                : num  -81.9 -81.9 -81.9 -81.9 -81.9 ...
\end{verbatim}

\hypertarget{wrangle-individual-datasets-to-create-processed-files.}{%
\subsection{Wrangle individual datasets to create processed
files.}\label{wrangle-individual-datasets-to-create-processed-files.}}

\begin{enumerate}
\def\labelenumi{\arabic{enumi}.}
\setcounter{enumi}{2}
\tightlist
\item
  Change date to date
\item
  Select the following columns: Date, DAILY\_AQI\_VALUE, Site.Name,
  AQS\_PARAMETER\_DESC, COUNTY, SITE\_LATITUDE, SITE\_LONGITUDE
\item
  For the PM2.5 datasets, fill all cells in AQS\_PARAMETER\_DESC with
  ``PM2.5'' (all cells in this column should be identical).
\item
  Save all four processed datasets in the Processed folder. Use the same
  file names as the raw files but replace ``raw'' with ``processed''.
\end{enumerate}

\hypertarget{combine-datasets}{%
\subsection{Combine datasets}\label{combine-datasets}}

\begin{enumerate}
\def\labelenumi{\arabic{enumi}.}
\setcounter{enumi}{6}
\tightlist
\item
  Combine the four datasets with \texttt{rbind}. Make sure your column
  names are identical prior to running this code.
\item
  Wrangle your new dataset with a pipe function (\%\textgreater{}\%) so
  that it fills the following conditions:
\end{enumerate}

\begin{itemize}
\tightlist
\item
  Include all sites that the four data frames have in common: ``Linville
  Falls'', ``Durham Armory'', ``Leggett'', ``Hattie Avenue'', ``Clemmons
  Middle'', ``Mendenhall School'', ``Frying Pan Mountain'', ``West
  Johnston Co.'', ``Garinger High School'', ``Castle Hayne'', ``Pitt
  Agri. Center'', ``Bryson City'', ``Millbrook School'' (the function
  \texttt{intersect} can figure out common factor levels)
\item
  Some sites have multiple measurements per day. Use the
  split-apply-combine strategy to generate daily means: group by date,
  site, aqs parameter, and county. Take the mean of the AQI value,
  latitude, and longitude.
\item
  Add columns for ``Month'' and ``Year'' by parsing your ``Date'' column
  (hint: \texttt{lubridate} package)
\item
  Hint: the dimensions of this dataset should be 14,752 x 9.
\end{itemize}

\begin{enumerate}
\def\labelenumi{\arabic{enumi}.}
\setcounter{enumi}{8}
\tightlist
\item
  Spread your datasets such that AQI values for ozone and PM2.5 are in
  separate columns. Each location on a specific date should now occupy
  only one row.
\item
  Call up the dimensions of your new tidy dataset.
\item
  Save your processed dataset with the following file name:
  ``EPAair\_O3\_PM25\_NC1718\_Processed.csv''
\end{enumerate}

\begin{Shaded}
\begin{Highlighting}[]
\CommentTok{#7}
\CommentTok{#rename column number 5 to match for all four datasets}
\KeywordTok{colnames}\NormalTok{(EPAAir1)[}\DecValTok{5}\NormalTok{] <-}\StringTok{ "Concentration"}
\KeywordTok{colnames}\NormalTok{(EPAAir2)[}\DecValTok{5}\NormalTok{] <-}\StringTok{ "Concentration"}
\KeywordTok{colnames}\NormalTok{(EPAAir25)[}\DecValTok{5}\NormalTok{] <-}\StringTok{ "Concentration"}
\KeywordTok{colnames}\NormalTok{(EPAAir25_}\DecValTok{2}\NormalTok{)[}\DecValTok{5}\NormalTok{] <-}\StringTok{ "Concentration"}

\CommentTok{#combine the four datasets using rbind}
\NormalTok{EPAAir.Datasets <-}\StringTok{ }\KeywordTok{rbind}\NormalTok{(EPAAir1, EPAAir2, EPAAir25, EPAAir25_}\DecValTok{2}\NormalTok{)}
\KeywordTok{view}\NormalTok{(EPAAir.Datasets)}

\CommentTok{#8}
\NormalTok{EPAAir.Datasets.Summaries <-}\StringTok{ }\KeywordTok{filter}\NormalTok{(EPAAir.Datasets, Site.Name }\OperatorTok\StringTok{ }\KeywordTok{c}\NormalTok{( }\StringTok{"Linville Falls"}\NormalTok{, }\StringTok{"Durham Armory"}\NormalTok{, }\StringTok{"Leggett"}\NormalTok{, }\StringTok{"Hattie Avenue"}\NormalTok{, }\StringTok{"Clemmons Middle"}\NormalTok{, }\StringTok{"Mendenhall School"}\NormalTok{, }\StringTok{"Frying Pan Mountain"}\NormalTok{, }\StringTok{"West Johnston Co."}\NormalTok{, }\StringTok{"Garinger High School"}\NormalTok{, }\StringTok{"Castle Hayne"}\NormalTok{, }\StringTok{"Pitt Agri. Center"}\NormalTok{, }\StringTok{"Bryson City"}\NormalTok{, }\StringTok{"Millbrook School"}\NormalTok{)) }\OperatorTok
\StringTok{  }\KeywordTok{group_by}\NormalTok{(Date, Site.Name, AQS_PARAMETER_DESC, COUNTY) }\OperatorTok
\StringTok{  }\KeywordTok{summarise}\NormalTok{(}\DataTypeTok{meanAQI =} \KeywordTok{mean}\NormalTok{(DAILY_AQI_VALUE),}
            \DataTypeTok{meanLAT =} \KeywordTok{mean}\NormalTok{(SITE_LATITUDE),}
            \DataTypeTok{meanLong =} \KeywordTok{mean}\NormalTok{(SITE_LONGITUDE)) }\OperatorTok
\StringTok{  }\KeywordTok{mutate}\NormalTok{(}\DataTypeTok{month =} \KeywordTok{month}\NormalTok{(Date)) }\OperatorTok
\StringTok{  }\KeywordTok{mutate}\NormalTok{(}\DataTypeTok{year =} \KeywordTok{year}\NormalTok{(Date))}

\CommentTok{#9}
\NormalTok{EPAAir.Datasets.Summaries.SPREAD <-}\StringTok{ }\KeywordTok{spread}\NormalTok{(EPAAir.Datasets.Summaries, AQS_PARAMETER_DESC, meanAQI)}

\CommentTok{#10 - Call the dimensions of the new tidy dataset}
\KeywordTok{dim}\NormalTok{(EPAAir.Datasets.Summaries.SPREAD)}
\end{Highlighting}
\end{Shaded}

\begin{verbatim}
## [1] 8976    9
\end{verbatim}

\begin{Shaded}
\begin{Highlighting}[]
\CommentTok{#11}
\KeywordTok{write.csv}\NormalTok{(EPAAir.Datasets.Summaries.SPREAD, }\DataTypeTok{row.names =} \OtherTok{FALSE}\NormalTok{,}
          \DataTypeTok{file =} \StringTok{"EPAair_O3_PM25_NC1718_Processed.csv"}\NormalTok{)}
\end{Highlighting}
\end{Shaded}

\hypertarget{generate-summary-tables}{%
\subsection{Generate summary tables}\label{generate-summary-tables}}

\begin{enumerate}
\def\labelenumi{\arabic{enumi}.}
\setcounter{enumi}{11}
\item
  Use the split-apply-combine strategy to generate a summary data frame.
  Data should be grouped by site, month, and year. Generate the mean AQI
  values for ozone and PM2.5 for each group. Then, add a pipe to remove
  instances where a month and year are not available (use the function
  \texttt{drop\_na} in your pipe).
\item
  Call up the dimensions of the summary dataset.
\end{enumerate}

\begin{Shaded}
\begin{Highlighting}[]
\CommentTok{#12a}
\NormalTok{EPAAir.Datasets.Summaries.dataframe <-}\StringTok{  }
\StringTok{  }\NormalTok{EPAAir.Datasets.Summaries.SPREAD }\OperatorTok\StringTok{ }
\StringTok{  }\KeywordTok{group_by}\NormalTok{(Site.Name, month, year) }\OperatorTok\StringTok{ }
\KeywordTok{summarise}\NormalTok{(}\DataTypeTok{mean0zone =} \KeywordTok{mean}\NormalTok{(Ozone), }\DataTypeTok{meanPM25 =} \KeywordTok{mean}\NormalTok{(PM2}\FloatTok{.5}\NormalTok{)) }\OperatorTok\StringTok{ }\KeywordTok{drop_na}\NormalTok{ (}\StringTok{"month"}\NormalTok{, }\StringTok{"year"}\NormalTok{)}


\CommentTok{#13}

\KeywordTok{dim}\NormalTok{(EPAAir.Datasets.Summaries.dataframe)}
\end{Highlighting}
\end{Shaded}

\begin{verbatim}
## [1] 308   5
\end{verbatim}

\begin{enumerate}
\def\labelenumi{\arabic{enumi}.}
\setcounter{enumi}{13}
\tightlist
\item
  Why did we use the function \texttt{drop\_na} rather than
  \texttt{na.omit}?
\end{enumerate}

\#\#Answer: Na omit removes all NAs (omit all rows containing missing
values), whereas drop\_na removes rows from a data frame when they have
missing values on a given variable.

\end{document}
