\PassOptionsToPackage{unicode=true}{hyperref} % options for packages loaded elsewhere
\PassOptionsToPackage{hyphens}{url}
%
\documentclass[]{article}
\usepackage{lmodern}
\usepackage{amssymb,amsmath}
\usepackage{ifxetex,ifluatex}
\usepackage{fixltx2e} % provides \textsubscript
\ifnum 0\ifxetex 1\fi\ifluatex 1\fi=0 % if pdftex
  \usepackage[T1]{fontenc}
  \usepackage[utf8]{inputenc}
  \usepackage{textcomp} % provides euro and other symbols
\else % if luatex or xelatex
  \usepackage{unicode-math}
  \defaultfontfeatures{Ligatures=TeX,Scale=MatchLowercase}
\fi
% use upquote if available, for straight quotes in verbatim environments
\IfFileExists{upquote.sty}{\usepackage{upquote}}{}
% use microtype if available
\IfFileExists{microtype.sty}{%
\usepackage[]{microtype}
\UseMicrotypeSet[protrusion]{basicmath} % disable protrusion for tt fonts
}{}
\IfFileExists{parskip.sty}{%
\usepackage{parskip}
}{% else
\setlength{\parindent}{0pt}
\setlength{\parskip}{6pt plus 2pt minus 1pt}
}
\usepackage{hyperref}
\hypersetup{
            pdftitle={Assignment 6: GLMs week 1 (t-test and ANOVA)},
            pdfauthor={Samantha Burch},
            pdfborder={0 0 0},
            breaklinks=true}
\urlstyle{same}  % don't use monospace font for urls
\usepackage[margin=2.54cm]{geometry}
\usepackage{color}
\usepackage{fancyvrb}
\newcommand{\VerbBar}{|}
\newcommand{\VERB}{\Verb[commandchars=\\\{\}]}
\DefineVerbatimEnvironment{Highlighting}{Verbatim}{commandchars=\\\{\}}
% Add ',fontsize=\small' for more characters per line
\usepackage{framed}
\definecolor{shadecolor}{RGB}{248,248,248}
\newenvironment{Shaded}{\begin{snugshade}}{\end{snugshade}}
\newcommand{\AlertTok}[1]{\textcolor[rgb]{0.94,0.16,0.16}{#1}}
\newcommand{\AnnotationTok}[1]{\textcolor[rgb]{0.56,0.35,0.01}{\textbf{\textit{#1}}}}
\newcommand{\AttributeTok}[1]{\textcolor[rgb]{0.77,0.63,0.00}{#1}}
\newcommand{\BaseNTok}[1]{\textcolor[rgb]{0.00,0.00,0.81}{#1}}
\newcommand{\BuiltInTok}[1]{#1}
\newcommand{\CharTok}[1]{\textcolor[rgb]{0.31,0.60,0.02}{#1}}
\newcommand{\CommentTok}[1]{\textcolor[rgb]{0.56,0.35,0.01}{\textit{#1}}}
\newcommand{\CommentVarTok}[1]{\textcolor[rgb]{0.56,0.35,0.01}{\textbf{\textit{#1}}}}
\newcommand{\ConstantTok}[1]{\textcolor[rgb]{0.00,0.00,0.00}{#1}}
\newcommand{\ControlFlowTok}[1]{\textcolor[rgb]{0.13,0.29,0.53}{\textbf{#1}}}
\newcommand{\DataTypeTok}[1]{\textcolor[rgb]{0.13,0.29,0.53}{#1}}
\newcommand{\DecValTok}[1]{\textcolor[rgb]{0.00,0.00,0.81}{#1}}
\newcommand{\DocumentationTok}[1]{\textcolor[rgb]{0.56,0.35,0.01}{\textbf{\textit{#1}}}}
\newcommand{\ErrorTok}[1]{\textcolor[rgb]{0.64,0.00,0.00}{\textbf{#1}}}
\newcommand{\ExtensionTok}[1]{#1}
\newcommand{\FloatTok}[1]{\textcolor[rgb]{0.00,0.00,0.81}{#1}}
\newcommand{\FunctionTok}[1]{\textcolor[rgb]{0.00,0.00,0.00}{#1}}
\newcommand{\ImportTok}[1]{#1}
\newcommand{\InformationTok}[1]{\textcolor[rgb]{0.56,0.35,0.01}{\textbf{\textit{#1}}}}
\newcommand{\KeywordTok}[1]{\textcolor[rgb]{0.13,0.29,0.53}{\textbf{#1}}}
\newcommand{\NormalTok}[1]{#1}
\newcommand{\OperatorTok}[1]{\textcolor[rgb]{0.81,0.36,0.00}{\textbf{#1}}}
\newcommand{\OtherTok}[1]{\textcolor[rgb]{0.56,0.35,0.01}{#1}}
\newcommand{\PreprocessorTok}[1]{\textcolor[rgb]{0.56,0.35,0.01}{\textit{#1}}}
\newcommand{\RegionMarkerTok}[1]{#1}
\newcommand{\SpecialCharTok}[1]{\textcolor[rgb]{0.00,0.00,0.00}{#1}}
\newcommand{\SpecialStringTok}[1]{\textcolor[rgb]{0.31,0.60,0.02}{#1}}
\newcommand{\StringTok}[1]{\textcolor[rgb]{0.31,0.60,0.02}{#1}}
\newcommand{\VariableTok}[1]{\textcolor[rgb]{0.00,0.00,0.00}{#1}}
\newcommand{\VerbatimStringTok}[1]{\textcolor[rgb]{0.31,0.60,0.02}{#1}}
\newcommand{\WarningTok}[1]{\textcolor[rgb]{0.56,0.35,0.01}{\textbf{\textit{#1}}}}
\usepackage{graphicx,grffile}
\makeatletter
\def\maxwidth{\ifdim\Gin@nat@width>\linewidth\linewidth\else\Gin@nat@width\fi}
\def\maxheight{\ifdim\Gin@nat@height>\textheight\textheight\else\Gin@nat@height\fi}
\makeatother
% Scale images if necessary, so that they will not overflow the page
% margins by default, and it is still possible to overwrite the defaults
% using explicit options in \includegraphics[width, height, ...]{}
\setkeys{Gin}{width=\maxwidth,height=\maxheight,keepaspectratio}
\setlength{\emergencystretch}{3em}  % prevent overfull lines
\providecommand{\tightlist}{%
  \setlength{\itemsep}{0pt}\setlength{\parskip}{0pt}}
\setcounter{secnumdepth}{0}
% Redefines (sub)paragraphs to behave more like sections
\ifx\paragraph\undefined\else
\let\oldparagraph\paragraph
\renewcommand{\paragraph}[1]{\oldparagraph{#1}\mbox{}}
\fi
\ifx\subparagraph\undefined\else
\let\oldsubparagraph\subparagraph
\renewcommand{\subparagraph}[1]{\oldsubparagraph{#1}\mbox{}}
\fi

% set default figure placement to htbp
\makeatletter
\def\fps@figure{htbp}
\makeatother


\title{Assignment 6: GLMs week 1 (t-test and ANOVA)}
\author{Samantha Burch}
\date{}

\begin{document}
\maketitle

\hypertarget{overview}{%
\subsection{OVERVIEW}\label{overview}}

This exercise accompanies the lessons in Environmental Data Analytics on
t-tests and ANOVAs.

\hypertarget{directions}{%
\subsection{Directions}\label{directions}}

\begin{enumerate}
\def\labelenumi{\arabic{enumi}.}
\tightlist
\item
  Change ``Student Name'' on line 3 (above) with your name.
\item
  Work through the steps, \textbf{creating code and output} that fulfill
  each instruction.
\item
  Be sure to \textbf{answer the questions} in this assignment document.
\item
  When you have completed the assignment, \textbf{Knit} the text and
  code into a single PDF file.
\item
  After Knitting, submit the completed exercise (PDF file) to the
  dropbox in Sakai. Add your last name into the file name (e.g.,
  ``Salk\_A06\_GLMs\_Week1.Rmd'') prior to submission.
\end{enumerate}

The completed exercise is due on Tuesday, February 18 at 1:00 pm.

\hypertarget{set-up-your-session}{%
\subsection{Set up your session}\label{set-up-your-session}}

\begin{enumerate}
\def\labelenumi{\arabic{enumi}.}
\item
  Check your working directory, load the \texttt{tidyverse},
  \texttt{cowplot}, and \texttt{agricolae} packages, and import the
  NTL-LTER\_Lake\_Nutrients\_PeterPaul\_Processed.csv dataset.
\item
  Change the date column to a date format. Call up \texttt{head} of this
  column to verify.
\end{enumerate}

\begin{Shaded}
\begin{Highlighting}[]
\CommentTok{#1}
\KeywordTok{getwd}\NormalTok{()}
\end{Highlighting}
\end{Shaded}

\begin{verbatim}
## [1] "/Users/samanthaburch/Desktop/Data Analytics/Environmental_Data_Analytics_2020"
\end{verbatim}

\begin{Shaded}
\begin{Highlighting}[]
\KeywordTok{library}\NormalTok{(tidyverse)}
\KeywordTok{library}\NormalTok{(cowplot)}
\KeywordTok{library}\NormalTok{(agricolae)}
\KeywordTok{library}\NormalTok{(ggplot2)}

\NormalTok{Litter <-}\StringTok{ }\KeywordTok{read.csv}\NormalTok{(}\StringTok{"./Data/Processed/NTL-LTER_Lake_Nutrients_PeterPaul_Processed.csv"}\NormalTok{)}

\CommentTok{#2}
\NormalTok{Litter}\OperatorTok{$}\NormalTok{sampledate <-}\StringTok{ }\KeywordTok{as.Date}\NormalTok{(Litter}\OperatorTok{$}\NormalTok{sampledate , }\DataTypeTok{format =} \StringTok{"%Y-%m-%d"}\NormalTok{)}
\KeywordTok{class}\NormalTok{(Litter}\OperatorTok{$}\NormalTok{sampledate)}
\end{Highlighting}
\end{Shaded}

\begin{verbatim}
## [1] "Date"
\end{verbatim}

\hypertarget{wrangle-your-data}{%
\subsection{Wrangle your data}\label{wrangle-your-data}}

\begin{enumerate}
\def\labelenumi{\arabic{enumi}.}
\setcounter{enumi}{2}
\tightlist
\item
  Wrangle your dataset so that it contains only surface depths and only
  the years 1993-1996, inclusive. Set month as a factor.
\end{enumerate}

\begin{Shaded}
\begin{Highlighting}[]
\NormalTok{Litter.Wrangle <-}\StringTok{ }\NormalTok{Litter }\OperatorTok\StringTok{ }
\StringTok{  }\KeywordTok{filter}\NormalTok{(depth }\OperatorTok{==}\StringTok{ }\DecValTok{0} \OperatorTok{&}\StringTok{ }\NormalTok{year4 }\OperatorTok{>}\StringTok{ }\DecValTok{1992} \OperatorTok{&}\StringTok{ }\NormalTok{year4 }\OperatorTok{<}\StringTok{ }\DecValTok{1997}\NormalTok{) }

\NormalTok{Litter.Wrangle}\OperatorTok{$}\NormalTok{month <-}\StringTok{ }\KeywordTok{as.factor}\NormalTok{(Litter.Wrangle}\OperatorTok{$}\NormalTok{month)}
\end{Highlighting}
\end{Shaded}

\hypertarget{analysis}{%
\subsection{Analysis}\label{analysis}}

Peter Lake was manipulated with additions of nitrogen and phosphorus
over the years 1993-1996 in an effort to assess the impacts of
eutrophication in lakes. You are tasked with finding out if nutrients
are significantly higher in Peter Lake than Paul Lake, and if these
potential differences in nutrients vary seasonally (use month as a
factor to represent seasonality). Run two separate tests for TN and TP.

\begin{enumerate}
\def\labelenumi{\arabic{enumi}.}
\setcounter{enumi}{3}
\tightlist
\item
  Which application of the GLM will you use (t-test, one-way ANOVA,
  two-way ANOVA with main effects, or two-way ANOVA with interaction
  effects)? Justify your choice.
\end{enumerate}

\begin{quote}
Answer: I would run a two-way ANOVA first and then with interaction
effects, because I will eventually want to examine both the individual
effects and interactions of the explanatory variables. Moreover, I want
to see if seasonality has an effect on nutrient levels of TN and TP.
\end{quote}

\begin{enumerate}
\def\labelenumi{\arabic{enumi}.}
\setcounter{enumi}{4}
\item
  Run your test for TN. Include examination of groupings and consider
  interaction effects, if relevant.
\item
  Run your test for TP. Include examination of groupings and consider
  interaction effects, if relevant.
\end{enumerate}

\begin{Shaded}
\begin{Highlighting}[]
\CommentTok{#5 TN, When we run aov for TN w/ lakename and month, we do not see a significant interaction between lakename:month...but we do with TP(note the asterisk/star)}
\NormalTok{Litter}\FloatTok{.2}\NormalTok{wayanova.TN <-}\StringTok{ }\KeywordTok{aov}\NormalTok{(}\DataTypeTok{data =}\NormalTok{ Litter.Wrangle, Litter.Wrangle}\OperatorTok{$}\NormalTok{tn_ug }\OperatorTok{~}\StringTok{ }\NormalTok{lakename }\OperatorTok{+}\StringTok{ }\NormalTok{month)}
\KeywordTok{summary}\NormalTok{(Litter}\FloatTok{.2}\NormalTok{wayanova.TN)}
\end{Highlighting}
\end{Shaded}

\begin{verbatim}
##              Df  Sum Sq Mean Sq F value   Pr(>F)    
## lakename      1 2468595 2468595   36.32 2.75e-08 ***
## month         4  459542  114885    1.69    0.158    
## Residuals   101 6864107   67961                     
## ---
## Signif. codes:  0 '***' 0.001 '**' 0.01 '*' 0.05 '.' 0.1 ' ' 1
## 23 observations deleted due to missingness
\end{verbatim}

\begin{Shaded}
\begin{Highlighting}[]
\KeywordTok{str}\NormalTok{(Litter.Wrangle)}
\end{Highlighting}
\end{Shaded}

\begin{verbatim}
## 'data.frame':    130 obs. of  14 variables:
##  $ lakeid    : Factor w/ 2 levels "L","R": 2 1 2 1 2 1 2 1 2 1 ...
##  $ lakename  : Factor w/ 2 levels "Paul Lake","Peter Lake": 2 1 2 1 2 1 2 1 2 1 ...
##  $ year4     : int  1993 1993 1993 1993 1993 1993 1993 1993 1993 1993 ...
##  $ daynum    : int  139 140 146 147 153 154 160 161 167 168 ...
##  $ month     : Factor w/ 5 levels "5","6","7","8",..: 1 1 1 1 2 2 2 2 2 2 ...
##  $ sampledate: Date, format: "1993-05-19" "1993-05-20" ...
##  $ depth_id  : int  1 1 1 1 1 1 1 1 1 1 ...
##  $ depth     : num  0 0 0 0 0 0 0 0 0 0 ...
##  $ tn_ug     : num  373 249 313 258 415 ...
##  $ tp_ug     : num  18.5 10.7 16.8 7 24.6 ...
##  $ nh34      : num  0 0 9.7 8.66 10.64 ...
##  $ no23      : num  5.5 1.15 9.93 3.21 11.54 ...
##  $ po4       : num  1 2 2 1 3 1 4 2 5 0 ...
##  $ comments  : logi  NA NA NA NA NA NA ...
\end{verbatim}

\begin{Shaded}
\begin{Highlighting}[]
\KeywordTok{TukeyHSD}\NormalTok{(Litter}\FloatTok{.2}\NormalTok{wayanova.TN)}
\end{Highlighting}
\end{Shaded}

\begin{verbatim}
##   Tukey multiple comparisons of means
##     95% family-wise confidence level
## 
## Fit: aov(formula = Litter.Wrangle$tn_ug ~ lakename + month, data = Litter.Wrangle)
## 
## $lakename
##                         diff      lwr      upr p adj
## Peter Lake-Paul Lake 303.796 203.8026 403.7894     0
## 
## $month
##          diff        lwr      upr     p adj
## 6-5 132.58168 -104.53533 369.6987 0.5307817
## 7-5 196.50011  -47.94924 440.9495 0.1761663
## 8-5 208.77984  -32.91447 450.4741 0.1238871
## 9-5 160.08048 -220.97835 541.1393 0.7701126
## 7-6  63.91843 -123.99128 251.8281 0.8785969
## 8-6  76.19815 -108.11330 260.5096 0.7803543
## 9-6  27.49879 -320.00718 375.0048 0.9994732
## 8-7  12.27972 -181.37388 205.9333 0.9997809
## 9-7 -36.41964 -388.96950 316.1302 0.9984948
## 9-8 -48.69936 -399.34457 301.9458 0.9952369
\end{verbatim}

\begin{Shaded}
\begin{Highlighting}[]
\CommentTok{#6 TP}
\NormalTok{Litter}\FloatTok{.2}\NormalTok{wayanova.TP <-}\StringTok{ }\KeywordTok{aov}\NormalTok{(}\DataTypeTok{data =}\NormalTok{ Litter.Wrangle, tp_ug }\OperatorTok{~}\StringTok{ }\NormalTok{lakename }\OperatorTok{+}\StringTok{ }\NormalTok{month)}
\KeywordTok{summary}\NormalTok{(Litter}\FloatTok{.2}\NormalTok{wayanova.TP)}
\end{Highlighting}
\end{Shaded}

\begin{verbatim}
##              Df Sum Sq Mean Sq F value Pr(>F)    
## lakename      1  10228   10228  94.453 <2e-16 ***
## month         4    813     203   1.876  0.119    
## Residuals   123  13320     108                   
## ---
## Signif. codes:  0 '***' 0.001 '**' 0.01 '*' 0.05 '.' 0.1 ' ' 1
## 1 observation deleted due to missingness
\end{verbatim}

\begin{Shaded}
\begin{Highlighting}[]
\KeywordTok{TukeyHSD}\NormalTok{(Litter}\FloatTok{.2}\NormalTok{wayanova.TP)}
\end{Highlighting}
\end{Shaded}

\begin{verbatim}
##   Tukey multiple comparisons of means
##     95% family-wise confidence level
## 
## Fit: aov(formula = tp_ug ~ lakename + month, data = Litter.Wrangle)
## 
## $lakename
##                          diff      lwr      upr p adj
## Peter Lake-Paul Lake 17.80939 14.18208 21.43669     0
## 
## $month
##           diff        lwr       upr     p adj
## 6-5  6.3451786  -3.012727 15.703084 0.3350273
## 7-5  8.8661326  -0.491773 18.224038 0.0723646
## 8-5  4.8191843  -4.469970 14.108339 0.6055077
## 9-5  5.4951391  -6.998304 17.988582 0.7410806
## 7-6  2.5209540  -4.366278  9.408186 0.8487741
## 8-6 -1.5259943  -8.319518  5.267530 0.9713266
## 9-6 -0.8500395 -11.618033  9.917954 0.9994865
## 8-7 -4.0469483 -10.840472  2.746576 0.4691480
## 9-7 -3.3709935 -14.138987  7.397000 0.9084852
## 9-8  0.6759548 -10.032345 11.384255 0.9997883
\end{verbatim}

\begin{Shaded}
\begin{Highlighting}[]
\CommentTok{#Interaction effects}
\NormalTok{Litter}\FloatTok{.2}\NormalTok{wayanova.TP.int <-}\StringTok{ }\KeywordTok{aov}\NormalTok{(}\DataTypeTok{data =}\NormalTok{ Litter.Wrangle, tp_ug }\OperatorTok{~}\StringTok{ }\NormalTok{lakename }\OperatorTok{*}\StringTok{ }\NormalTok{month) }\CommentTok{#only thing that changes is multiplication vs plus sign}
\KeywordTok{summary}\NormalTok{(Litter}\FloatTok{.2}\NormalTok{wayanova.TP.int)}
\end{Highlighting}
\end{Shaded}

\begin{verbatim}
##                 Df Sum Sq Mean Sq F value Pr(>F)    
## lakename         1  10228   10228  98.914 <2e-16 ***
## month            4    813     203   1.965 0.1043    
## lakename:month   4   1014     254   2.452 0.0496 *  
## Residuals      119  12305     103                   
## ---
## Signif. codes:  0 '***' 0.001 '**' 0.01 '*' 0.05 '.' 0.1 ' ' 1
## 1 observation deleted due to missingness
\end{verbatim}

\begin{Shaded}
\begin{Highlighting}[]
\NormalTok{TP.int <-}\StringTok{ }\KeywordTok{with}\NormalTok{(Litter.Wrangle, }\KeywordTok{interaction}\NormalTok{(lakename, month))}
\NormalTok{TP.anova.int <-}\StringTok{ }\KeywordTok{aov}\NormalTok{(}\DataTypeTok{data =}\NormalTok{ Litter.Wrangle, tp_ug }\OperatorTok{~}\StringTok{ }\NormalTok{TP.int)}

\NormalTok{TP.groups <-}\StringTok{ }\KeywordTok{HSD.test}\NormalTok{(TP.anova.int, }\StringTok{"TP.int"}\NormalTok{, }\DataTypeTok{group =} \OtherTok{TRUE}\NormalTok{) }\CommentTok{#we do this bc it is significant, p value is less than .05}

\NormalTok{Litter}\FloatTok{.2}\NormalTok{wayanova.TN.int <-}\StringTok{ }\KeywordTok{aov}\NormalTok{(}\DataTypeTok{data =}\NormalTok{ Litter.Wrangle, tn_ug }\OperatorTok{~}\StringTok{ }\NormalTok{lakename }\OperatorTok{*}\StringTok{ }\NormalTok{month) }\CommentTok{#only thing that changes is multiplication vs plus sign}
\KeywordTok{summary}\NormalTok{(Litter}\FloatTok{.2}\NormalTok{wayanova.TN.int)}
\end{Highlighting}
\end{Shaded}

\begin{verbatim}
##                Df  Sum Sq Mean Sq F value   Pr(>F)    
## lakename        1 2468595 2468595  36.414 2.91e-08 ***
## month           4  459542  114885   1.695    0.157    
## lakename:month  4  288272   72068   1.063    0.379    
## Residuals      97 6575834   67792                     
## ---
## Signif. codes:  0 '***' 0.001 '**' 0.01 '*' 0.05 '.' 0.1 ' ' 1
## 23 observations deleted due to missingness
\end{verbatim}

\begin{enumerate}
\def\labelenumi{\arabic{enumi}.}
\setcounter{enumi}{6}
\item
  Create two plots, with TN (plot 1) or TP (plot 2) as the response
  variable and month and lake as the predictor variables. Hint: you may
  use some of the code you used for your visualization assignment.
  Assign groupings with letters, as determined from your tests. Adjust
  your axes, aesthetics, and color palettes in accordance with best data
  visualization practices.
\item
  Combine your plots with cowplot, with a common legend at the top and
  the two graphs stacked vertically. Your x axes should be formatted
  with the same breaks, such that you can remove the title and text of
  the top legend and retain just the bottom legend.
\end{enumerate}

\begin{Shaded}
\begin{Highlighting}[]
\CommentTok{#7.1 TN boxplot}
\NormalTok{TNplot <-}
\StringTok{    }\KeywordTok{ggplot}\NormalTok{(Litter.Wrangle, }\KeywordTok{aes}\NormalTok{(}\DataTypeTok{y =}\NormalTok{ tn_ug, }\DataTypeTok{x =} \KeywordTok{as.factor}\NormalTok{(month), }\DataTypeTok{color =}\NormalTok{ lakename)) }\OperatorTok{+}
\StringTok{  }\KeywordTok{geom_boxplot}\NormalTok{() }\OperatorTok{+}
\StringTok{  }\KeywordTok{labs}\NormalTok{(}\DataTypeTok{x =} \KeywordTok{expression}\NormalTok{(}\KeywordTok{paste}\NormalTok{(}\StringTok{"Month"}\NormalTok{)),}
       \DataTypeTok{y =} \KeywordTok{expression}\NormalTok{(}\KeywordTok{paste}\NormalTok{(}\StringTok{"TN"} \OperatorTok{~}\StringTok{ }\NormalTok{(mu}\OperatorTok{*}\NormalTok{g }\OperatorTok{/}\StringTok{ }\NormalTok{L)))) }\OperatorTok{+}
\StringTok{  }\KeywordTok{scale_color_brewer}\NormalTok{(}\DataTypeTok{palette =} \StringTok{"Dark2"}\NormalTok{) }\OperatorTok{+}
\StringTok{  }\KeywordTok{ylim}\NormalTok{(}\DecValTok{0}\NormalTok{,}\DecValTok{2500}\NormalTok{) }\OperatorTok{+}
\StringTok{  }\KeywordTok{stat_summary}\NormalTok{(}\DataTypeTok{geom =} \StringTok{"text"}\NormalTok{, }\DataTypeTok{fun.y =}\NormalTok{ max, }\DataTypeTok{vjust =} \DecValTok{-1}\NormalTok{, }\DataTypeTok{size =} \DecValTok{4}\NormalTok{, }\DataTypeTok{position =} \KeywordTok{position_dodge}\NormalTok{(.}\DecValTok{7}\NormalTok{), }\DataTypeTok{label =} \KeywordTok{c}\NormalTok{(}\StringTok{"a"}\NormalTok{, }\StringTok{"b"}\NormalTok{, }\StringTok{"a"}\NormalTok{, }\StringTok{"b"}\NormalTok{, }\StringTok{"a"}\NormalTok{, }\StringTok{"b"}\NormalTok{, }\StringTok{"a"}\NormalTok{, }\StringTok{"b"}\NormalTok{, }\StringTok{"a"}\NormalTok{, }\StringTok{"b"}\NormalTok{)) }\OperatorTok{+}
\StringTok{  }\KeywordTok{theme}\NormalTok{(}\DataTypeTok{legend.position =} \StringTok{"none"}\NormalTok{)}
\KeywordTok{print}\NormalTok{(TNplot)}
\end{Highlighting}
\end{Shaded}

\begin{verbatim}
## Warning: Removed 23 rows containing non-finite values (stat_boxplot).
\end{verbatim}

\begin{verbatim}
## Warning: Removed 23 rows containing non-finite values (stat_summary).
\end{verbatim}

\includegraphics{A06_GLMs_Week1_files/figure-latex/unnamed-chunk-4-1.pdf}

\begin{Shaded}
\begin{Highlighting}[]
\CommentTok{#7.2 TP boxplot}
\NormalTok{TPplot <-}
\StringTok{    }\KeywordTok{ggplot}\NormalTok{(Litter.Wrangle, }\KeywordTok{aes}\NormalTok{(}\DataTypeTok{y =}\NormalTok{ tp_ug, }\DataTypeTok{x =} \KeywordTok{as.factor}\NormalTok{(month), }\DataTypeTok{color =}\NormalTok{ lakename)) }\OperatorTok{+}
\StringTok{  }\KeywordTok{geom_boxplot}\NormalTok{() }\OperatorTok{+}
\StringTok{  }\KeywordTok{labs}\NormalTok{(}\DataTypeTok{x =} \KeywordTok{expression}\NormalTok{(}\KeywordTok{paste}\NormalTok{(}\StringTok{"Month"}\NormalTok{)),}
       \DataTypeTok{y =} \KeywordTok{expression}\NormalTok{(}\KeywordTok{paste}\NormalTok{(}\StringTok{"TP"} \OperatorTok{~}\StringTok{ }\NormalTok{(mu}\OperatorTok{*}\NormalTok{g }\OperatorTok{/}\StringTok{ }\NormalTok{L)))) }\OperatorTok{+}
\StringTok{  }\KeywordTok{scale_color_brewer}\NormalTok{(}\DataTypeTok{palette =} \StringTok{"Dark2"}\NormalTok{) }\OperatorTok{+}
\StringTok{  }\KeywordTok{ylim}\NormalTok{(}\DecValTok{0}\NormalTok{,}\DecValTok{80}\NormalTok{) }\OperatorTok{+}
\StringTok{  }\KeywordTok{stat_summary}\NormalTok{(}\DataTypeTok{geom =} \StringTok{"text"}\NormalTok{, }\DataTypeTok{fun.y =}\NormalTok{ max, }\DataTypeTok{vjust =} \DecValTok{-1}\NormalTok{, }\DataTypeTok{size =} \DecValTok{4}\NormalTok{, }\DataTypeTok{position =} \KeywordTok{position_dodge}\NormalTok{(.}\DecValTok{7}\NormalTok{), }\DataTypeTok{label =} \KeywordTok{c}\NormalTok{(}\StringTok{"bcd"}\NormalTok{, }\StringTok{"cd"}\NormalTok{, }\StringTok{"d"}\NormalTok{, }\StringTok{"ab"}\NormalTok{, }\StringTok{"a"}\NormalTok{, }\StringTok{"d"}\NormalTok{, }\StringTok{"abc"}\NormalTok{, }\StringTok{"d"}\NormalTok{, }\StringTok{"abcd"}\NormalTok{, }\StringTok{"cd"}\NormalTok{)) }\OperatorTok{+}
\StringTok{  }\KeywordTok{theme}\NormalTok{(}\DataTypeTok{legend.position =} \StringTok{"bottom"}\NormalTok{,}
        \DataTypeTok{legend.text =} \KeywordTok{element_text}\NormalTok{(}\DataTypeTok{size =} \DecValTok{12}\NormalTok{), }\DataTypeTok{legend.title =} \KeywordTok{element_text}\NormalTok{(}\DataTypeTok{size =} \DecValTok{12}\NormalTok{))}
\KeywordTok{print}\NormalTok{(TPplot)}
\end{Highlighting}
\end{Shaded}

\begin{verbatim}
## Warning: Removed 1 rows containing non-finite values (stat_boxplot).
\end{verbatim}

\begin{verbatim}
## Warning: Removed 1 rows containing non-finite values (stat_summary).
\end{verbatim}

\includegraphics{A06_GLMs_Week1_files/figure-latex/unnamed-chunk-4-2.pdf}

\begin{Shaded}
\begin{Highlighting}[]
\CommentTok{#8}
\NormalTok{TP_TN_plots <-}\StringTok{ }\KeywordTok{plot_grid}\NormalTok{(TNplot, TPplot,}
                         \DataTypeTok{align =} \StringTok{"vh"}\NormalTok{, }\DataTypeTok{ncol =} \DecValTok{1}\NormalTok{)}
\end{Highlighting}
\end{Shaded}

\begin{verbatim}
## Warning: Removed 23 rows containing non-finite values (stat_boxplot).
\end{verbatim}

\begin{verbatim}
## Warning: Removed 23 rows containing non-finite values (stat_summary).
\end{verbatim}

\begin{verbatim}
## Warning: Removed 1 rows containing non-finite values (stat_boxplot).
\end{verbatim}

\begin{verbatim}
## Warning: Removed 1 rows containing non-finite values (stat_summary).
\end{verbatim}

\begin{verbatim}
## Warning: Graphs cannot be horizontally aligned unless the axis parameter is set.
## Placing graphs unaligned.
\end{verbatim}

\begin{Shaded}
\begin{Highlighting}[]
\KeywordTok{print}\NormalTok{(TP_TN_plots)}
\end{Highlighting}
\end{Shaded}

\includegraphics{A06_GLMs_Week1_files/figure-latex/unnamed-chunk-4-3.pdf}

\end{document}
