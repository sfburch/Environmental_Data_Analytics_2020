\PassOptionsToPackage{unicode=true}{hyperref} % options for packages loaded elsewhere
\PassOptionsToPackage{hyphens}{url}
%
\documentclass[]{article}
\usepackage{lmodern}
\usepackage{amssymb,amsmath}
\usepackage{ifxetex,ifluatex}
\usepackage{fixltx2e} % provides \textsubscript
\ifnum 0\ifxetex 1\fi\ifluatex 1\fi=0 % if pdftex
  \usepackage[T1]{fontenc}
  \usepackage[utf8]{inputenc}
  \usepackage{textcomp} % provides euro and other symbols
\else % if luatex or xelatex
  \usepackage{unicode-math}
  \defaultfontfeatures{Ligatures=TeX,Scale=MatchLowercase}
\fi
% use upquote if available, for straight quotes in verbatim environments
\IfFileExists{upquote.sty}{\usepackage{upquote}}{}
% use microtype if available
\IfFileExists{microtype.sty}{%
\usepackage[]{microtype}
\UseMicrotypeSet[protrusion]{basicmath} % disable protrusion for tt fonts
}{}
\IfFileExists{parskip.sty}{%
\usepackage{parskip}
}{% else
\setlength{\parindent}{0pt}
\setlength{\parskip}{6pt plus 2pt minus 1pt}
}
\usepackage{hyperref}
\hypersetup{
            pdftitle={Assignment 3: Data Exploration},
            pdfauthor={Samantha Burch},
            pdfborder={0 0 0},
            breaklinks=true}
\urlstyle{same}  % don't use monospace font for urls
\usepackage[margin=2.54cm]{geometry}
\usepackage{color}
\usepackage{fancyvrb}
\newcommand{\VerbBar}{|}
\newcommand{\VERB}{\Verb[commandchars=\\\{\}]}
\DefineVerbatimEnvironment{Highlighting}{Verbatim}{commandchars=\\\{\}}
% Add ',fontsize=\small' for more characters per line
\usepackage{framed}
\definecolor{shadecolor}{RGB}{248,248,248}
\newenvironment{Shaded}{\begin{snugshade}}{\end{snugshade}}
\newcommand{\AlertTok}[1]{\textcolor[rgb]{0.94,0.16,0.16}{#1}}
\newcommand{\AnnotationTok}[1]{\textcolor[rgb]{0.56,0.35,0.01}{\textbf{\textit{#1}}}}
\newcommand{\AttributeTok}[1]{\textcolor[rgb]{0.77,0.63,0.00}{#1}}
\newcommand{\BaseNTok}[1]{\textcolor[rgb]{0.00,0.00,0.81}{#1}}
\newcommand{\BuiltInTok}[1]{#1}
\newcommand{\CharTok}[1]{\textcolor[rgb]{0.31,0.60,0.02}{#1}}
\newcommand{\CommentTok}[1]{\textcolor[rgb]{0.56,0.35,0.01}{\textit{#1}}}
\newcommand{\CommentVarTok}[1]{\textcolor[rgb]{0.56,0.35,0.01}{\textbf{\textit{#1}}}}
\newcommand{\ConstantTok}[1]{\textcolor[rgb]{0.00,0.00,0.00}{#1}}
\newcommand{\ControlFlowTok}[1]{\textcolor[rgb]{0.13,0.29,0.53}{\textbf{#1}}}
\newcommand{\DataTypeTok}[1]{\textcolor[rgb]{0.13,0.29,0.53}{#1}}
\newcommand{\DecValTok}[1]{\textcolor[rgb]{0.00,0.00,0.81}{#1}}
\newcommand{\DocumentationTok}[1]{\textcolor[rgb]{0.56,0.35,0.01}{\textbf{\textit{#1}}}}
\newcommand{\ErrorTok}[1]{\textcolor[rgb]{0.64,0.00,0.00}{\textbf{#1}}}
\newcommand{\ExtensionTok}[1]{#1}
\newcommand{\FloatTok}[1]{\textcolor[rgb]{0.00,0.00,0.81}{#1}}
\newcommand{\FunctionTok}[1]{\textcolor[rgb]{0.00,0.00,0.00}{#1}}
\newcommand{\ImportTok}[1]{#1}
\newcommand{\InformationTok}[1]{\textcolor[rgb]{0.56,0.35,0.01}{\textbf{\textit{#1}}}}
\newcommand{\KeywordTok}[1]{\textcolor[rgb]{0.13,0.29,0.53}{\textbf{#1}}}
\newcommand{\NormalTok}[1]{#1}
\newcommand{\OperatorTok}[1]{\textcolor[rgb]{0.81,0.36,0.00}{\textbf{#1}}}
\newcommand{\OtherTok}[1]{\textcolor[rgb]{0.56,0.35,0.01}{#1}}
\newcommand{\PreprocessorTok}[1]{\textcolor[rgb]{0.56,0.35,0.01}{\textit{#1}}}
\newcommand{\RegionMarkerTok}[1]{#1}
\newcommand{\SpecialCharTok}[1]{\textcolor[rgb]{0.00,0.00,0.00}{#1}}
\newcommand{\SpecialStringTok}[1]{\textcolor[rgb]{0.31,0.60,0.02}{#1}}
\newcommand{\StringTok}[1]{\textcolor[rgb]{0.31,0.60,0.02}{#1}}
\newcommand{\VariableTok}[1]{\textcolor[rgb]{0.00,0.00,0.00}{#1}}
\newcommand{\VerbatimStringTok}[1]{\textcolor[rgb]{0.31,0.60,0.02}{#1}}
\newcommand{\WarningTok}[1]{\textcolor[rgb]{0.56,0.35,0.01}{\textbf{\textit{#1}}}}
\usepackage{graphicx,grffile}
\makeatletter
\def\maxwidth{\ifdim\Gin@nat@width>\linewidth\linewidth\else\Gin@nat@width\fi}
\def\maxheight{\ifdim\Gin@nat@height>\textheight\textheight\else\Gin@nat@height\fi}
\makeatother
% Scale images if necessary, so that they will not overflow the page
% margins by default, and it is still possible to overwrite the defaults
% using explicit options in \includegraphics[width, height, ...]{}
\setkeys{Gin}{width=\maxwidth,height=\maxheight,keepaspectratio}
\setlength{\emergencystretch}{3em}  % prevent overfull lines
\providecommand{\tightlist}{%
  \setlength{\itemsep}{0pt}\setlength{\parskip}{0pt}}
\setcounter{secnumdepth}{0}
% Redefines (sub)paragraphs to behave more like sections
\ifx\paragraph\undefined\else
\let\oldparagraph\paragraph
\renewcommand{\paragraph}[1]{\oldparagraph{#1}\mbox{}}
\fi
\ifx\subparagraph\undefined\else
\let\oldsubparagraph\subparagraph
\renewcommand{\subparagraph}[1]{\oldsubparagraph{#1}\mbox{}}
\fi

% set default figure placement to htbp
\makeatletter
\def\fps@figure{htbp}
\makeatother


\title{Assignment 3: Data Exploration}
\author{Samantha Burch}
\date{}

\begin{document}
\maketitle

\hypertarget{overview}{%
\subsection{OVERVIEW}\label{overview}}

This exercise accompanies the lessons in Environmental Data Analytics on
Data Exploration.

\hypertarget{directions}{%
\subsection{Directions}\label{directions}}

\begin{enumerate}
\def\labelenumi{\arabic{enumi}.}
\tightlist
\item
  Change ``Student Name'' on line 3 (above) with your name.
\item
  Work through the steps, \textbf{creating code and output} that fulfill
  each instruction.
\item
  Be sure to \textbf{answer the questions} in this assignment document.
\item
  When you have completed the assignment, \textbf{Knit} the text and
  code into a single PDF file.
\item
  After Knitting, submit the completed exercise (PDF file) to the
  dropbox in Sakai. Add your last name into the file name (e.g.,
  ``Salk\_A03\_DataExploration.Rmd'') prior to submission.
\end{enumerate}

The completed exercise is due on Tuesday, January 28 at 1:00 pm.

\hypertarget{set-up-your-r-session}{%
\subsection{Set up your R session}\label{set-up-your-r-session}}

\begin{enumerate}
\def\labelenumi{\arabic{enumi}.}
\tightlist
\item
  Check your working directory, load necessary packages (tidyverse), and
  upload two datasets: the ECOTOX neonicotinoid dataset
  (ECOTOX\_Neonicotinoids\_Insects\_raw.csv) and the Niwot Ridge NEON
  dataset for litter and woody debris
  (NEON\_NIWO\_Litter\_massdata\_2018-08\_raw.csv). Name these datasets
  ``Neonics'' and ``Litter'', respectively.
\end{enumerate}

\begin{Shaded}
\begin{Highlighting}[]
\KeywordTok{getwd}\NormalTok{()}
\end{Highlighting}
\end{Shaded}

\begin{verbatim}
## [1] "/Users/samanthaburch/Desktop/Data Analytics/Environmental_Data_Analytics_2020"
\end{verbatim}

\begin{Shaded}
\begin{Highlighting}[]
\CommentTok{#Load packages}
\KeywordTok{library}\NormalTok{(tidyverse)}

\CommentTok{#Import data}
\NormalTok{Litter <-}\StringTok{ }\KeywordTok{read.csv}\NormalTok{(}\StringTok{"./Data/Raw/NEON_NIWO_Litter_massdata_2018-08_raw.csv"}\NormalTok{)}
\NormalTok{Neonics <-}\StringTok{ }\KeywordTok{read.csv}\NormalTok{(}\StringTok{"./Data/Raw/ECOTOX_Neonicotinoids_Insects_raw.csv"}\NormalTok{)}
\end{Highlighting}
\end{Shaded}

\hypertarget{learn-about-your-system}{%
\subsection{Learn about your system}\label{learn-about-your-system}}

\begin{enumerate}
\def\labelenumi{\arabic{enumi}.}
\setcounter{enumi}{1}
\tightlist
\item
  The neonicotinoid dataset was collected from the Environmental
  Protection Agency's ECOTOX Knowledgebase, a database for ecotoxicology
  research. Neonicotinoids are a class of insecticides used widely in
  agriculture. The dataset that has been pulled includes all studies
  published on insects. Why might we be interested in the ecotoxicologoy
  of neonicotinoids on insects? Feel free to do a brief internet search
  if you feel you need more background information.
\end{enumerate}

\begin{quote}
Answer: Neonicotoids are considered to be highly effective insecticides
for the following: 1) crop protection against pests, and 2) flea control
for both cats and dogs. These widly spread/used insecticides could be
posing a threat to aquatic environments, as they first contaminate the
soil on which they're used and then their residues trickle down into our
water systems. To date, there is little awareness of the impacts of
Neonicotoids on aquatic environments and ecosystems overall (how much
are absorbed by plants that insects feed on); thus, it is important to
close such mentioned knowledge gaps by further analyzing data linked to
its use.(\url{https://www.ncbi.nlm.nih.gov/pubmed/15822177}) These
insecticides need to be studied further to understand their effect on
insects (i.e.~bees).
\end{quote}

\begin{enumerate}
\def\labelenumi{\arabic{enumi}.}
\setcounter{enumi}{2}
\tightlist
\item
  The Niwot Ridge litter and woody debris dataset was collected from the
  National Ecological Observatory Network, which collectively includes
  81 aquatic and terrestrial sites across 20 ecoclimatic domains. 32 of
  these sites sample forest litter and woody debris, and we will focus
  on the Niwot Ridge long-term ecological research (LTER) station in
  Colorado. Why might we be interested in studying litter and woody
  debris that falls to the ground in forests? Feel free to do a brief
  internet search if you feel you need more background information.
\end{enumerate}

\begin{quote}
Answer: It is important to study this as such litter and woody debris
can impact forest communities and negatively impact ground-dwelling
invertebrates (i.e.~via forest fire risk). These types of disturbances
can alter habitat structure, energy and nutrient flow, and ultimately
shape critical ecosystem processes.
(\url{https://www.mdpi.com/1999-4907/8/5/174/htm})
\end{quote}

\begin{enumerate}
\def\labelenumi{\arabic{enumi}.}
\setcounter{enumi}{3}
\tightlist
\item
  How is litter and woody debris sampled as part of the NEON network?
  Read the NEON\_Litterfall\_UserGuide.pdf document to learn more. List
  three pieces of salient information about the sampling methods here:
\end{enumerate}

\begin{quote}
Answer: * Each collection event is measured separately for different
functional groups (i.e.~leaves, twigs, needles) * All masses are
reported ``at the spatial resolution of a single trap and the temporal
resolution of a single collection event.'' No single site should have
more than 3,440 data instances in a single calendar year. * It is
important to pay attention to data relationships and ensure to check the
data for anomolies before joining tables. Lastly, in order to provide
context to litter data, users are encouraged to leverage data from
vegetation structure.
\end{quote}

\hypertarget{obtain-basic-summaries-of-your-data-neonics}{%
\subsection{Obtain basic summaries of your data
(Neonics)}\label{obtain-basic-summaries-of-your-data-neonics}}

\begin{enumerate}
\def\labelenumi{\arabic{enumi}.}
\setcounter{enumi}{4}
\tightlist
\item
  What are the dimensions of the dataset?
\end{enumerate}

\begin{Shaded}
\begin{Highlighting}[]
\KeywordTok{dim}\NormalTok{(Neonics)}
\end{Highlighting}
\end{Shaded}

\begin{verbatim}
## [1] 4623   30
\end{verbatim}

\begin{enumerate}
\def\labelenumi{\arabic{enumi}.}
\setcounter{enumi}{5}
\tightlist
\item
  Using the \texttt{summary} function, determine the most common effects
  that are studied. Why might these effects specifically be of interest?
\end{enumerate}

\begin{Shaded}
\begin{Highlighting}[]
\KeywordTok{summary}\NormalTok{(Neonics)}
\end{Highlighting}
\end{Shaded}

\begin{verbatim}
##    CAS.Number       
##  Min.   : 58842209  
##  1st Qu.:138261413  
##  Median :138261413  
##  Mean   :147651982  
##  3rd Qu.:153719234  
##  Max.   :210880925  
##                     
##                                                                                 Chemical.Name 
##  (2E)-1-[(6-Chloro-3-pyridinyl)methyl]-N-nitro-2-imidazolidinimine                     :2658  
##  3-[(2-Chloro-5-thiazolyl)methyl]tetrahydro-5-methyl-N-nitro-4H-1,3,5-oxadiazin-4-imine: 686  
##  [C(E)]-N-[(2-Chloro-5-thiazolyl)methyl]-N'-methyl-N''-nitroguanidine                  : 452  
##  (1E)-N-[(6-Chloro-3-pyridinyl)methyl]-N'-cyano-N-methylethanimidamide                 : 420  
##  N''-Methyl-N-nitro-N'-[(tetrahydro-3-furanyl)methyl]guanidine                         : 218  
##  [N(Z)]-N-[3-[(6-Chloro-3-pyridinyl)methyl]-2-thiazolidinylidene]cyanamide             : 128  
##  (Other)                                                                               :  61  
##                                                    Chemical.Grade
##  Not reported                                             :3989  
##  Technical grade, technical product, technical formulation: 422  
##  Pestanal grade                                           :  93  
##  Not coded                                                :  53  
##  Commercial grade                                         :  27  
##  Analytical grade                                         :  15  
##  (Other)                                                  :  24  
##                                                  Chemical.Analysis.Method
##  Measured                                                    : 230       
##  Not coded                                                   :  51       
##  Not reported                                                :   5       
##  Unmeasured                                                  :4321       
##  Unmeasured values (some measured values reported in article):  16       
##                                                                          
##                                                                          
##  Chemical.Purity                  Species.Scientific.Name
##  NR     :2502    Apis mellifera               : 667      
##  25     : 244    Bombus terrestris            : 183      
##  50     : 200    Apis mellifera ssp. carnica  : 152      
##  20     : 189    Bombus impatiens             : 140      
##  70     : 112    Apis mellifera ssp. ligustica: 113      
##  75     :  89    Popillia japonica            :  94      
##  (Other):1287    (Other)                      :3274      
##             Species.Common.Name
##  Honey Bee            : 667    
##  Parasitic Wasp       : 285    
##  Buff Tailed Bumblebee: 183    
##  Carniolan Honey Bee  : 152    
##  Bumble Bee           : 140    
##  Italian Honeybee     : 113    
##  (Other)              :3083    
##                                                        Species.Group 
##  Insects/Spiders                                              :3569  
##  Insects/Spiders; Standard Test Species                       :  27  
##  Insects/Spiders; Standard Test Species; U.S. Invasive Species: 667  
##  Insects/Spiders; U.S. Invasive Species                       : 360  
##                                                                      
##                                                                      
##                                                                      
##     Organism.Lifestage  Organism.Age             Organism.Age.Units
##  Not reported:2271     NR     :3851   Not reported        :3515    
##  Adult       :1222     2      : 111   Day(s)              : 327    
##  Larva       : 437     3      : 105   Instar              : 255    
##  Multiple    : 285     <24    :  81   Hour(s)             : 241    
##  Egg         : 128     4      :  81   Hours post-emergence:  99    
##  Pupa        :  69     1      :  59   Year(s)             :  64    
##  (Other)     : 211     (Other): 335   (Other)             : 122    
##                     Exposure.Type         Media.Type  
##  Environmental, unspecified:1599   No substrate:2934  
##  Food                      :1124   Not reported: 663  
##  Spray                     : 393   Natural soil: 393  
##  Topical, general          : 254   Litter      : 264  
##  Ground granular           : 249   Filter paper: 230  
##  Hand spray                : 210   Not coded   :  51  
##  (Other)                   : 794   (Other)     :  88  
##               Test.Location  Number.of.Doses        Conc.1.Type..Author.
##  Field artificial    :  96   2      :2441    Active ingredient:3161     
##  Field natural       :1663   3      : 499    Formulation      :1420     
##  Field undeterminable:   4   5      : 314    Not coded        :  42     
##  Lab                 :2860   6      : 230                               
##                              4      : 221                               
##                              NR     : 217                               
##                              (Other): 701                               
##  Conc.1..Author. Conc.1.Units..Author.              Effect    
##  0.37/  : 208    AI kg/ha  : 575       Population      :1803  
##  10/    : 127    AI mg/L   : 298       Mortality       :1493  
##  NR/    : 108    AI lb/acre: 277       Behavior        : 360  
##  NR     :  94    AI g/ha   : 241       Feeding behavior: 255  
##  1      :  82    ng/org    : 231       Reproduction    : 197  
##  1023   :  80    ppm       : 180       Development     : 136  
##  (Other):3924    (Other)   :2821       (Other)         : 379  
##               Effect.Measurement    Endpoint                   Response.Site 
##  Abundance             :1699     NOEL   :1816   Not reported          :4349  
##  Mortality             :1294     LOEL   :1664   Midgut or midgut gland:  63  
##  Survival              : 133     LC50   : 327   Not coded             :  51  
##  Progeny counts/numbers: 120     LD50   : 274   Whole organism        :  41  
##  Food consumption      : 103     NR     : 167   Hypopharyngeal gland  :  27  
##  Emergence             :  98     NR-LETH:  86   Head                  :  23  
##  (Other)               :1176     (Other): 289   (Other)               :  69  
##  Observed.Duration..Days.       Observed.Duration.Units..Days.
##  1      : 713             Day(s)               :4394          
##  2      : 383             Emergence            :  70          
##  NR     : 355             Growing season       :  48          
##  7      : 207             Day(s) post-hatch    :  20          
##  3      : 183             Day(s) post-emergence:  17          
##  0.0417 : 133             Tiller stage         :  15          
##  (Other):2649             (Other)              :  59          
##                                                                            Author    
##  Peck,D.C.                                                                    : 208  
##  Frank,S.D.                                                                   : 100  
##  El Hassani,A.K., M. Dacher, V. Gary, M. Lambin, M. Gauthier, and C. Armengaud:  96  
##  Williamson,S.M., S.J. Willis, and G.A. Wright                                :  93  
##  Laurino,D., A. Manino, A. Patetta, and M. Porporato                          :  88  
##  Scholer,J., and V. Krischik                                                  :  82  
##  (Other)                                                                      :3956  
##  Reference.Number
##  Min.   :   344  
##  1st Qu.:108459  
##  Median :165559  
##  Mean   :142189  
##  3rd Qu.:168998  
##  Max.   :180410  
##                  
##                                                                                                                                         Title     
##  Long-Term Effects of Imidacloprid on the Abundance of Surface- and Soil-Active Nontarget Fauna in Turf                                    : 200  
##  Reduced Risk Insecticides to Control Scale Insects and Protect Natural Enemies in the Production and Maintenance of Urban Landscape Plants: 100  
##  Effects of Sublethal Doses of Acetamiprid and Thiamethoxam on the Behavior of the Honeybee (Apis mellifera)                               :  96  
##  Exposure to Neonicotinoids Influences the Motor Function of Adult Worker Honeybees                                                        :  93  
##  Toxicity of Neonicotinoid Insecticides on Different Honey Bee Genotypes                                                                   :  88  
##  Chronic Exposure of Imidacloprid and Clothianidin Reduce Queen Survival, Foraging, and Nectar Storing in Colonies of Bombus impatiens     :  82  
##  (Other)                                                                                                                                   :3964  
##                                            Source     Publication.Year
##  Agric. For. Entomol.11(4): 405-419           : 200   Min.   :1982    
##  Environ. Entomol.41(2): 377-386              : 100   1st Qu.:2005    
##  Arch. Environ. Contam. Toxicol.54(4): 653-661:  96   Median :2010    
##  Ecotoxicology23:1409-1418                    :  93   Mean   :2008    
##  Bull. Insectol.66(1): 119-126                :  88   3rd Qu.:2013    
##  PLoS One9(3): 14 p.                          :  82   Max.   :2019    
##  (Other)                                      :3964                   
##  Summary.of.Additional.Parameters                                                                                                                                                                                                                       
##  Purity: \xca NR - NR | Organism Age: \xca NR - NR Not reported | Conc 1 (Author): \xca Active ingredient NR/ - NR/ AI kg/ha | Duration (Days): \xca NR - NR NR | Conc 2 (Author): \xca NR (NR - NR) NR | Conc 3 (Author): \xca NR (NR - NR) NR  : 389  
##  Purity: \xca NR - NR | Organism Age: \xca NR - NR Not reported | Conc 1 (Author): \xca Active ingredient NR - NR AI lb/acre | Duration (Days): \xca NR - NR NR | Conc 2 (Author): \xca NR (NR - NR) NR | Conc 3 (Author): \xca NR (NR - NR) NR  : 138  
##  Purity: \xca NR - NR | Organism Age: \xca NR - NR Not reported | Conc 1 (Author): \xca Active ingredient NR - NR AI kg/ha | Duration (Days): \xca NR - NR NR | Conc 2 (Author): \xca NR (NR - NR) NR | Conc 3 (Author): \xca NR (NR - NR) NR    : 136  
##  Purity: \xca NR - NR | Organism Age: \xca NR - NR Not reported | Conc 1 (Author): \xca Active ingredient NR/ - NR/ AI lb/acre | Duration (Days): \xca NR - NR NR | Conc 2 (Author): \xca NR (NR - NR) NR | Conc 3 (Author): \xca NR (NR - NR) NR: 124  
##  Purity: \xca NR - NR | Organism Age: \xca NR - NR Not reported | Conc 1 (Author): \xca Active ingredient NR - NR AI ng/org | Duration (Days): \xca NR - NR NR | Conc 2 (Author): \xca NR (NR - NR) NR | Conc 3 (Author): \xca NR (NR - NR) NR   :  94  
##  Purity: \xca NR - NR | Organism Age: \xca NR - NR Not reported | Conc 1 (Author): \xca Formulation NR - NR ml/ha | Duration (Days): \xca NR - NR NR | Conc 2 (Author): \xca NR (NR - NR) NR | Conc 3 (Author): \xca NR (NR - NR) NR             :  80  
##  (Other)                                                                                                                                                                                                                                         :3662
\end{verbatim}

\begin{quote}
Answer: It is critical to understand if there are any discrepancies in,
say, how long an organism was observed and its specific exposure type.
This also allows you to see important summary statistics in one place,
including for example: mean, median, and quartiles. The most commonly
studied effects include abundance and mortality, which can help
researchers better understand insect populations at different points in
their life.
\end{quote}

\begin{enumerate}
\def\labelenumi{\arabic{enumi}.}
\setcounter{enumi}{6}
\tightlist
\item
  Using the \texttt{summary} function, determine the six most commonly
  studied species in the dataset (common name). What do these species
  have in common, and why might they be of interest over other insects?
  Feel free to do a brief internet search for more information if
  needed.
\end{enumerate}

\begin{Shaded}
\begin{Highlighting}[]
\KeywordTok{summary}\NormalTok{(Neonics)}
\end{Highlighting}
\end{Shaded}

\begin{verbatim}
##    CAS.Number       
##  Min.   : 58842209  
##  1st Qu.:138261413  
##  Median :138261413  
##  Mean   :147651982  
##  3rd Qu.:153719234  
##  Max.   :210880925  
##                     
##                                                                                 Chemical.Name 
##  (2E)-1-[(6-Chloro-3-pyridinyl)methyl]-N-nitro-2-imidazolidinimine                     :2658  
##  3-[(2-Chloro-5-thiazolyl)methyl]tetrahydro-5-methyl-N-nitro-4H-1,3,5-oxadiazin-4-imine: 686  
##  [C(E)]-N-[(2-Chloro-5-thiazolyl)methyl]-N'-methyl-N''-nitroguanidine                  : 452  
##  (1E)-N-[(6-Chloro-3-pyridinyl)methyl]-N'-cyano-N-methylethanimidamide                 : 420  
##  N''-Methyl-N-nitro-N'-[(tetrahydro-3-furanyl)methyl]guanidine                         : 218  
##  [N(Z)]-N-[3-[(6-Chloro-3-pyridinyl)methyl]-2-thiazolidinylidene]cyanamide             : 128  
##  (Other)                                                                               :  61  
##                                                    Chemical.Grade
##  Not reported                                             :3989  
##  Technical grade, technical product, technical formulation: 422  
##  Pestanal grade                                           :  93  
##  Not coded                                                :  53  
##  Commercial grade                                         :  27  
##  Analytical grade                                         :  15  
##  (Other)                                                  :  24  
##                                                  Chemical.Analysis.Method
##  Measured                                                    : 230       
##  Not coded                                                   :  51       
##  Not reported                                                :   5       
##  Unmeasured                                                  :4321       
##  Unmeasured values (some measured values reported in article):  16       
##                                                                          
##                                                                          
##  Chemical.Purity                  Species.Scientific.Name
##  NR     :2502    Apis mellifera               : 667      
##  25     : 244    Bombus terrestris            : 183      
##  50     : 200    Apis mellifera ssp. carnica  : 152      
##  20     : 189    Bombus impatiens             : 140      
##  70     : 112    Apis mellifera ssp. ligustica: 113      
##  75     :  89    Popillia japonica            :  94      
##  (Other):1287    (Other)                      :3274      
##             Species.Common.Name
##  Honey Bee            : 667    
##  Parasitic Wasp       : 285    
##  Buff Tailed Bumblebee: 183    
##  Carniolan Honey Bee  : 152    
##  Bumble Bee           : 140    
##  Italian Honeybee     : 113    
##  (Other)              :3083    
##                                                        Species.Group 
##  Insects/Spiders                                              :3569  
##  Insects/Spiders; Standard Test Species                       :  27  
##  Insects/Spiders; Standard Test Species; U.S. Invasive Species: 667  
##  Insects/Spiders; U.S. Invasive Species                       : 360  
##                                                                      
##                                                                      
##                                                                      
##     Organism.Lifestage  Organism.Age             Organism.Age.Units
##  Not reported:2271     NR     :3851   Not reported        :3515    
##  Adult       :1222     2      : 111   Day(s)              : 327    
##  Larva       : 437     3      : 105   Instar              : 255    
##  Multiple    : 285     <24    :  81   Hour(s)             : 241    
##  Egg         : 128     4      :  81   Hours post-emergence:  99    
##  Pupa        :  69     1      :  59   Year(s)             :  64    
##  (Other)     : 211     (Other): 335   (Other)             : 122    
##                     Exposure.Type         Media.Type  
##  Environmental, unspecified:1599   No substrate:2934  
##  Food                      :1124   Not reported: 663  
##  Spray                     : 393   Natural soil: 393  
##  Topical, general          : 254   Litter      : 264  
##  Ground granular           : 249   Filter paper: 230  
##  Hand spray                : 210   Not coded   :  51  
##  (Other)                   : 794   (Other)     :  88  
##               Test.Location  Number.of.Doses        Conc.1.Type..Author.
##  Field artificial    :  96   2      :2441    Active ingredient:3161     
##  Field natural       :1663   3      : 499    Formulation      :1420     
##  Field undeterminable:   4   5      : 314    Not coded        :  42     
##  Lab                 :2860   6      : 230                               
##                              4      : 221                               
##                              NR     : 217                               
##                              (Other): 701                               
##  Conc.1..Author. Conc.1.Units..Author.              Effect    
##  0.37/  : 208    AI kg/ha  : 575       Population      :1803  
##  10/    : 127    AI mg/L   : 298       Mortality       :1493  
##  NR/    : 108    AI lb/acre: 277       Behavior        : 360  
##  NR     :  94    AI g/ha   : 241       Feeding behavior: 255  
##  1      :  82    ng/org    : 231       Reproduction    : 197  
##  1023   :  80    ppm       : 180       Development     : 136  
##  (Other):3924    (Other)   :2821       (Other)         : 379  
##               Effect.Measurement    Endpoint                   Response.Site 
##  Abundance             :1699     NOEL   :1816   Not reported          :4349  
##  Mortality             :1294     LOEL   :1664   Midgut or midgut gland:  63  
##  Survival              : 133     LC50   : 327   Not coded             :  51  
##  Progeny counts/numbers: 120     LD50   : 274   Whole organism        :  41  
##  Food consumption      : 103     NR     : 167   Hypopharyngeal gland  :  27  
##  Emergence             :  98     NR-LETH:  86   Head                  :  23  
##  (Other)               :1176     (Other): 289   (Other)               :  69  
##  Observed.Duration..Days.       Observed.Duration.Units..Days.
##  1      : 713             Day(s)               :4394          
##  2      : 383             Emergence            :  70          
##  NR     : 355             Growing season       :  48          
##  7      : 207             Day(s) post-hatch    :  20          
##  3      : 183             Day(s) post-emergence:  17          
##  0.0417 : 133             Tiller stage         :  15          
##  (Other):2649             (Other)              :  59          
##                                                                            Author    
##  Peck,D.C.                                                                    : 208  
##  Frank,S.D.                                                                   : 100  
##  El Hassani,A.K., M. Dacher, V. Gary, M. Lambin, M. Gauthier, and C. Armengaud:  96  
##  Williamson,S.M., S.J. Willis, and G.A. Wright                                :  93  
##  Laurino,D., A. Manino, A. Patetta, and M. Porporato                          :  88  
##  Scholer,J., and V. Krischik                                                  :  82  
##  (Other)                                                                      :3956  
##  Reference.Number
##  Min.   :   344  
##  1st Qu.:108459  
##  Median :165559  
##  Mean   :142189  
##  3rd Qu.:168998  
##  Max.   :180410  
##                  
##                                                                                                                                         Title     
##  Long-Term Effects of Imidacloprid on the Abundance of Surface- and Soil-Active Nontarget Fauna in Turf                                    : 200  
##  Reduced Risk Insecticides to Control Scale Insects and Protect Natural Enemies in the Production and Maintenance of Urban Landscape Plants: 100  
##  Effects of Sublethal Doses of Acetamiprid and Thiamethoxam on the Behavior of the Honeybee (Apis mellifera)                               :  96  
##  Exposure to Neonicotinoids Influences the Motor Function of Adult Worker Honeybees                                                        :  93  
##  Toxicity of Neonicotinoid Insecticides on Different Honey Bee Genotypes                                                                   :  88  
##  Chronic Exposure of Imidacloprid and Clothianidin Reduce Queen Survival, Foraging, and Nectar Storing in Colonies of Bombus impatiens     :  82  
##  (Other)                                                                                                                                   :3964  
##                                            Source     Publication.Year
##  Agric. For. Entomol.11(4): 405-419           : 200   Min.   :1982    
##  Environ. Entomol.41(2): 377-386              : 100   1st Qu.:2005    
##  Arch. Environ. Contam. Toxicol.54(4): 653-661:  96   Median :2010    
##  Ecotoxicology23:1409-1418                    :  93   Mean   :2008    
##  Bull. Insectol.66(1): 119-126                :  88   3rd Qu.:2013    
##  PLoS One9(3): 14 p.                          :  82   Max.   :2019    
##  (Other)                                      :3964                   
##  Summary.of.Additional.Parameters                                                                                                                                                                                                                       
##  Purity: \xca NR - NR | Organism Age: \xca NR - NR Not reported | Conc 1 (Author): \xca Active ingredient NR/ - NR/ AI kg/ha | Duration (Days): \xca NR - NR NR | Conc 2 (Author): \xca NR (NR - NR) NR | Conc 3 (Author): \xca NR (NR - NR) NR  : 389  
##  Purity: \xca NR - NR | Organism Age: \xca NR - NR Not reported | Conc 1 (Author): \xca Active ingredient NR - NR AI lb/acre | Duration (Days): \xca NR - NR NR | Conc 2 (Author): \xca NR (NR - NR) NR | Conc 3 (Author): \xca NR (NR - NR) NR  : 138  
##  Purity: \xca NR - NR | Organism Age: \xca NR - NR Not reported | Conc 1 (Author): \xca Active ingredient NR - NR AI kg/ha | Duration (Days): \xca NR - NR NR | Conc 2 (Author): \xca NR (NR - NR) NR | Conc 3 (Author): \xca NR (NR - NR) NR    : 136  
##  Purity: \xca NR - NR | Organism Age: \xca NR - NR Not reported | Conc 1 (Author): \xca Active ingredient NR/ - NR/ AI lb/acre | Duration (Days): \xca NR - NR NR | Conc 2 (Author): \xca NR (NR - NR) NR | Conc 3 (Author): \xca NR (NR - NR) NR: 124  
##  Purity: \xca NR - NR | Organism Age: \xca NR - NR Not reported | Conc 1 (Author): \xca Active ingredient NR - NR AI ng/org | Duration (Days): \xca NR - NR NR | Conc 2 (Author): \xca NR (NR - NR) NR | Conc 3 (Author): \xca NR (NR - NR) NR   :  94  
##  Purity: \xca NR - NR | Organism Age: \xca NR - NR Not reported | Conc 1 (Author): \xca Formulation NR - NR ml/ha | Duration (Days): \xca NR - NR NR | Conc 2 (Author): \xca NR (NR - NR) NR | Conc 3 (Author): \xca NR (NR - NR) NR             :  80  
##  (Other)                                                                                                                                                                                                                                         :3662
\end{verbatim}

\begin{quote}
Answer: In this dataset, the six most commonly studied species include:
1) Honeybee, 2) Parasitic Wasp, 3) Carniolan Honeybee, 4) Bumble Bee,
and 6) Italian Honeybee. These mentioned species can be categorized into
`pollinator insects,' which indicates that they are key to the vitality
of our ecosystems.
\end{quote}

\begin{enumerate}
\def\labelenumi{\arabic{enumi}.}
\setcounter{enumi}{7}
\tightlist
\item
  Concentrations are always a numeric value. What is the class of
  Conc.1..Author. in the dataset, and why is it not numeric?
\end{enumerate}

\begin{Shaded}
\begin{Highlighting}[]
\KeywordTok{class}\NormalTok{(}\StringTok{"Con.1.Author"}\NormalTok{)}
\end{Highlighting}
\end{Shaded}

\begin{verbatim}
## [1] "character"
\end{verbatim}

\begin{Shaded}
\begin{Highlighting}[]
\KeywordTok{class}\NormalTok{(Neonics}\OperatorTok{$}\NormalTok{Conc.}\DecValTok{1}\NormalTok{..Author.)}
\end{Highlighting}
\end{Shaded}

\begin{verbatim}
## [1] "factor"
\end{verbatim}

\begin{quote}
Answer: This is considered to be a character class, but then when I
re-ran it using the Neonics dataset, I received `factor' as the output.
It is not numeric because it is listed as an active ingredient.
\end{quote}

\hypertarget{explore-your-data-graphically-neonics}{%
\subsection{Explore your data graphically
(Neonics)}\label{explore-your-data-graphically-neonics}}

\begin{enumerate}
\def\labelenumi{\arabic{enumi}.}
\setcounter{enumi}{8}
\tightlist
\item
  Using \texttt{geom\_freqpoly}, generate a plot of the number of
  studies conducted by publication year.
\end{enumerate}

\begin{Shaded}
\begin{Highlighting}[]
\KeywordTok{ggplot}\NormalTok{(Neonics)}
\end{Highlighting}
\end{Shaded}

\includegraphics{A03_DataExploration_files/figure-latex/unnamed-chunk-6-1.pdf}

\begin{Shaded}
\begin{Highlighting}[]
\KeywordTok{ggplot}\NormalTok{(Neonics) }\OperatorTok{+}
\StringTok{  }\KeywordTok{geom_freqpoly}\NormalTok{(}\KeywordTok{aes}\NormalTok{(}\DataTypeTok{x =}\NormalTok{ Publication.Year))}
\end{Highlighting}
\end{Shaded}

\begin{verbatim}
## `stat_bin()` using `bins = 30`. Pick better value with `binwidth`.
\end{verbatim}

\includegraphics{A03_DataExploration_files/figure-latex/unnamed-chunk-6-2.pdf}

\begin{enumerate}
\def\labelenumi{\arabic{enumi}.}
\setcounter{enumi}{9}
\tightlist
\item
  Reproduce the same graph but now add a color aesthetic so that
  different Test.Location are displayed as different colors.
\end{enumerate}

\begin{Shaded}
\begin{Highlighting}[]
\KeywordTok{ggplot}\NormalTok{(Neonics) }\OperatorTok{+}
\StringTok{  }\KeywordTok{geom_freqpoly}\NormalTok{(}\KeywordTok{aes}\NormalTok{(}\DataTypeTok{x =}\NormalTok{ Publication.Year, }\DataTypeTok{binwidth =} \DecValTok{15}\NormalTok{, }\DataTypeTok{color =}\NormalTok{ Test.Location))}
\end{Highlighting}
\end{Shaded}

\begin{verbatim}
## Warning: Ignoring unknown aesthetics: binwidth
\end{verbatim}

\begin{verbatim}
## `stat_bin()` using `bins = 30`. Pick better value with `binwidth`.
\end{verbatim}

\includegraphics{A03_DataExploration_files/figure-latex/unnamed-chunk-7-1.pdf}

Interpret this graph. What are the most common test locations, and do
they differ over time?

\begin{quote}
Answer: The two most common publications and test locations are via 1)
the lab, and 2) field natural. These generally peak and dip at the same
time; however, there is a large jump in lab use between about 2013 and
2015. This could denote years in which there were an increased number of
publications and lab space became more readily available for conducting
tests.
\end{quote}

\begin{enumerate}
\def\labelenumi{\arabic{enumi}.}
\setcounter{enumi}{10}
\tightlist
\item
  Create a bar graph of Endpoint counts. What are the two most common
  end points, and how are they defined? Consult the ECOTOX\_CodeAppendix
  for more information.
\end{enumerate}

\begin{Shaded}
\begin{Highlighting}[]
\KeywordTok{ggplot}\NormalTok{(Neonics, }\KeywordTok{aes}\NormalTok{(}\DataTypeTok{x =}\NormalTok{ Endpoint)) }\OperatorTok{+}
\StringTok{  }\KeywordTok{geom_bar}\NormalTok{()}
\end{Highlighting}
\end{Shaded}

\includegraphics{A03_DataExploration_files/figure-latex/unnamed-chunk-8-1.pdf}

\begin{Shaded}
\begin{Highlighting}[]
\KeywordTok{class}\NormalTok{(}\StringTok{"Endpoint"}\NormalTok{)}
\end{Highlighting}
\end{Shaded}

\begin{verbatim}
## [1] "character"
\end{verbatim}

\begin{quote}
Answer: The two most common endpoints are `NOEL' and `LOEL,' and they
are defined as characters. LOEL is the Lowest-Observable-Effect-Level
and is considered the lowest dose concentration. NOEL stands for
No-Observable-Effect-Level and is considered the highest dose
concentration; this does not produce significantly different effects
from the response controls.
\end{quote}

\hypertarget{explore-your-data-litter}{%
\subsection{Explore your data (Litter)}\label{explore-your-data-litter}}

\begin{enumerate}
\def\labelenumi{\arabic{enumi}.}
\setcounter{enumi}{11}
\tightlist
\item
  Determine the class of collectDate. Is it a date? If not, change to a
  date and confirm the new class of the variable. Using the
  \texttt{unique} function, determine which dates litter was sampled in
  August 2018.
\end{enumerate}

\begin{Shaded}
\begin{Highlighting}[]
\KeywordTok{class}\NormalTok{(}\StringTok{"collectDate"}\NormalTok{) }\CommentTok{#character}
\end{Highlighting}
\end{Shaded}

\begin{verbatim}
## [1] "character"
\end{verbatim}

\begin{Shaded}
\begin{Highlighting}[]
\KeywordTok{class}\NormalTok{(Litter}\OperatorTok{$}\NormalTok{datetime)}
\end{Highlighting}
\end{Shaded}

\begin{verbatim}
## [1] "NULL"
\end{verbatim}

\begin{Shaded}
\begin{Highlighting}[]
\NormalTok{Litter}\OperatorTok{$}\NormalTok{datetime <-}\StringTok{ }\KeywordTok{format}\NormalTok{(Litter}\OperatorTok{$}\NormalTok{datetime, }\StringTok{"%y"}\NormalTok{)}
\NormalTok{Litter}\OperatorTok{$}\NormalTok{collectDate <-}\StringTok{ }\KeywordTok{as.Date}\NormalTok{(Litter}\OperatorTok{$}\NormalTok{collectDate, }\DataTypeTok{format =} \StringTok{"%Y-%m-%d"}\NormalTok{)}
\KeywordTok{class}\NormalTok{(Litter}\OperatorTok{$}\NormalTok{collectDate) }\CommentTok{#factor #date}
\end{Highlighting}
\end{Shaded}

\begin{verbatim}
## [1] "Date"
\end{verbatim}

\begin{Shaded}
\begin{Highlighting}[]
\KeywordTok{unique}\NormalTok{(Litter}\OperatorTok{$}\NormalTok{collectDate)}
\end{Highlighting}
\end{Shaded}

\begin{verbatim}
## [1] "2018-08-02" "2018-08-30"
\end{verbatim}

\begin{Shaded}
\begin{Highlighting}[]
\KeywordTok{unique}\NormalTok{(Litter[,}\StringTok{"collectDate"}\NormalTok{])}
\end{Highlighting}
\end{Shaded}

\begin{verbatim}
## [1] "2018-08-02" "2018-08-30"
\end{verbatim}

\begin{enumerate}
\def\labelenumi{\arabic{enumi}.}
\setcounter{enumi}{12}
\tightlist
\item
  Using the \texttt{unique} function, determine how many plots were
  sampled at Niwot Ridge. How is the information obtained from
  \texttt{unique} different from that obtained from \texttt{summary}?
\end{enumerate}

\begin{Shaded}
\begin{Highlighting}[]
\KeywordTok{unique}\NormalTok{(Litter}\OperatorTok{$}\NormalTok{plotID) }\CommentTok{#12}
\end{Highlighting}
\end{Shaded}

\begin{verbatim}
##  [1] NIWO_061 NIWO_064 NIWO_067 NIWO_040 NIWO_041 NIWO_063 NIWO_047 NIWO_051
##  [9] NIWO_058 NIWO_046 NIWO_062 NIWO_057
## 12 Levels: NIWO_040 NIWO_041 NIWO_046 NIWO_047 NIWO_051 NIWO_057 ... NIWO_067
\end{verbatim}

\begin{quote}
Answer: The summary function is considered a generic function that
produces summaries of the results of ``various model fitting
functions.'' Where as the unique function is used to return a vector,
data frame or array with any duplicate rows/elements removed (will be
useful for pipes). This shows distinct factors.
\end{quote}

\begin{enumerate}
\def\labelenumi{\arabic{enumi}.}
\setcounter{enumi}{13}
\tightlist
\item
  Create a bar graph of functionalGroup counts. This shows you what type
  of litter is collected at the Niwot Ridge sites. Notice that litter
  types are fairly equally distributed across the Niwot Ridge sites.
\end{enumerate}

\begin{Shaded}
\begin{Highlighting}[]
\KeywordTok{ggplot}\NormalTok{(Litter, }\KeywordTok{aes}\NormalTok{(}\DataTypeTok{x =}\NormalTok{ functionalGroup)) }\OperatorTok{+}
\StringTok{  }\KeywordTok{geom_bar}\NormalTok{()}
\end{Highlighting}
\end{Shaded}

\includegraphics{A03_DataExploration_files/figure-latex/unnamed-chunk-11-1.pdf}

\begin{enumerate}
\def\labelenumi{\arabic{enumi}.}
\setcounter{enumi}{14}
\tightlist
\item
  Using \texttt{geom\_boxplot} and \texttt{geom\_violin}, create a
  boxplot and a violin plot of dryMass by functionalGroup.
\end{enumerate}

\begin{Shaded}
\begin{Highlighting}[]
\CommentTok{#geom boxplot}
\KeywordTok{ggplot}\NormalTok{(Litter) }\OperatorTok{+}
\StringTok{  }\KeywordTok{geom_boxplot}\NormalTok{(}\KeywordTok{aes}\NormalTok{(}\DataTypeTok{x =}\NormalTok{ functionalGroup, }\DataTypeTok{y =}\NormalTok{ dryMass, }\DataTypeTok{group =} \KeywordTok{cut_width}\NormalTok{(functionalGroup, }\DecValTok{1}\NormalTok{)))}
\end{Highlighting}
\end{Shaded}

\includegraphics{A03_DataExploration_files/figure-latex/unnamed-chunk-12-1.pdf}

\begin{Shaded}
\begin{Highlighting}[]
\CommentTok{#Violin plot}
\KeywordTok{ggplot}\NormalTok{(Litter) }\OperatorTok{+}
\StringTok{  }\KeywordTok{geom_violin}\NormalTok{(}\KeywordTok{aes}\NormalTok{(}\DataTypeTok{x =}\NormalTok{ functionalGroup, }\DataTypeTok{y =}\NormalTok{ dryMass),}
                  \DataTypeTok{draw_quantiles =} \KeywordTok{c}\NormalTok{(.}\DecValTok{25}\NormalTok{, }\FloatTok{.5}\NormalTok{, }\FloatTok{.75}\NormalTok{),}
              \DataTypeTok{scale =} \StringTok{"count"}\NormalTok{)}
\end{Highlighting}
\end{Shaded}

\begin{verbatim}
## Warning in regularize.values(x, y, ties, missing(ties)): collapsing to unique
## 'x' values

## Warning in regularize.values(x, y, ties, missing(ties)): collapsing to unique
## 'x' values

## Warning in regularize.values(x, y, ties, missing(ties)): collapsing to unique
## 'x' values
\end{verbatim}

\includegraphics{A03_DataExploration_files/figure-latex/unnamed-chunk-12-2.pdf}

Why is the boxplot a more effective visualization option than the violin
plot in this case?

\begin{quote}
Answer: The violin plot does not allow one to see any outliers nor the
middle portion (50\%) of the whole data distribution, whereas the
boxplot does. It is difficult to understand what the violin plot is
communicating.
\end{quote}

What type(s) of litter tend to have the highest biomass at these sites?

\begin{quote}
Answer: Needles have the highest biomass at these sites.
\end{quote}

\end{document}
